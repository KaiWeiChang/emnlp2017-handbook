While language identification works well on standard texts, it performs much worse on social media language, in particular dialectal language---even for English. First, to support work on English language identification, we contribute a new dataset of tweets annotated for English versus non-English, with attention to ambiguity, code-switching, and automatic generation issues. It is randomly sampled from all public messages, avoiding biases towards pre-existing language classifiers. Second, we find that a demographic language model---which identifies messages with language similar to that used by several U.S. ethnic populations on Twitter---can be used to improve English language identification performance when combined with a traditional supervised language identifier. It increases recall with almost no loss of precision, including, surprisingly, for English messages written by non-U.S. authors. Our dataset and identifier ensemble are available online.
