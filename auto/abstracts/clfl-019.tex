We study perceptions of literariness in a set of contemporary Dutch novels. Experiments with machine learning models show that it is possible to automatically distinguish novels that are seen as highly literary from those that are seen as less literary, using surprisingly simple textual features. The most discriminating features of our classification model indicate that genre might be a confounding factor, but a regression model shows that we can also explain variation between highly literary novels from less literary ones within genre.
