Previous research on word embeddings has shown that sparse representations, which can be either learned on top of existing dense embeddings or obtained through model constraints during training time, have the benefit of increased interpretability properties: to some degree, each dimension can be understood by a human and associated with a recognizable feature in the data. In this paper, we transfer this idea to sentence embeddings and explore several approaches to obtain a sparse representation. We further introduce a novel, quantitative and automated evaluation metric for sentence embedding interpretability, based on topic coherence methods. We observe an increase in interpretability compared to dense models, on a dataset of movie dialogs and on the scene descriptions from the MS COCO dataset.
