An important skill in critical thinking and argumentation is the ability to spot and recognize fallacies. Fallacious arguments, omnipresent in argumentative discourse, can be deceptive, manipulative, or simply leading to 'wrong moves' in a discussion. Despite their importance, argumentation scholars and NLP researchers with focus on argumentation quality have not yet investigated fallacies empirically. The nonexistence of resources dealing with fallacious argumentation calls for scalable approaches to data acquisition and annotation, for which the serious games methodology offers an appealing, yet unexplored, alternative. We present Argotario, a serious game that deals with fallacies in everyday argumentation. Argotario is a multilingual, open-source, platform-independent application with strong educational aspects, accessible at www.argotario.net.
