The importance of balancing linguistic considerations, annotation practicalities, and end user needs in developing language annotation guidelines is discussed. Maintaining a clear view of the various goals and fostering collaboration and feedback across levels of annotation and between corpus creators and corpus users is helpful in determining this balance. Annotating non-canonical language brings additional challenges that serve to highlight the necessity of keeping these goals in mind when creating corpora.
