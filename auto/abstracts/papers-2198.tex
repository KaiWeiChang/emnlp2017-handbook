Amidst growing concern over media manipulation, NLP attention has focused  on overt strategies like censorship and ``fake news''. Here, we draw on two concepts from political science literature to explore subtler strategies for government media manipulation: agenda-setting (selecting what topics to cover) and framing (deciding how topics are covered). We analyze 13 years (100K articles) of the Russian newspaper Izvestia and identify a strategy of distraction: articles mention the U.S. more frequently in the month directly following an economic downturn in Russia. We introduce embedding-based methods for cross-lingually projecting English frames to Russian, and discover that these articles emphasize U.S. moral failings and threats to the U.S. Our work offers new ways to identify subtle media manipulation strategies at the intersection of agenda-setting and framing.
