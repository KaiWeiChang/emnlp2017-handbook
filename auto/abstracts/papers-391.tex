We pose the general task of user-factor adaptation -- adapting supervised learning models to real-valued user factors inferred from a background of their language, reflecting the idea that a piece of text should be understood within the context of the user that wrote it. We introduce a continuous adaptation technique, suited for real-valued user factors that are common in social science and bringing us closer to personalized NLP, adapting to each user uniquely. We apply this technique with known user factors including age, gender, and personality traits, as well as latent factors, evaluating over five tasks: POS tagging, PP-attachment, sentiment analysis, sarcasm detection, and stance detection. Adaptation provides statistically significant benefits for 3 of the 5 tasks: up to +1.2 points for PP-attachment, +3.4 points for sarcasm, and +3.0 points for stance.
