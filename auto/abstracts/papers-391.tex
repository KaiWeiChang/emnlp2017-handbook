Convolutional neural networks (CNNs) have recently emerged as a popular building block for natural language processing (NLP). Despite their success, most existing CNN models employed in NLP share the  same learned (and static) set of filters for all input sentences. In this paper, we consider an approach of using a small meta network to learn context-sensitive convolutional filters for text processing. The role of meta network is to abstract the contextual information of a sentence or document into a set of input-sensitive filters. We further generalize this framework to model sentence pairs, where a bidirectional filter generation mechanism is introduced to encapsulate co-dependent sentence representations.  In our benchmarks on four different tasks,  including ontology classification, sentiment analysis, answer sentence selection, and paraphrase identification, our proposed model, a modified CNN with context-sensitive filters, consistently outperforms the standard CNN and attention-based CNN baselines. By visualizing the learned context-sensitive filters, we further validate and rationalize the effectiveness of proposed framework.
