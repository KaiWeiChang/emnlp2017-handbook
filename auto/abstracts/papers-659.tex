This article evaluates three proposed laws of semantic change. Our claim is that in order to validate a putative law of semantic change, the effect should be observed in the genuine condition but absent or reduced in a suitably matched control condition, in which no change can possibly have taken place. Our analysis shows that the effects reported in recent literature must be substantially revised: (i) the proposed negative correlation between meaning change and word frequency is shown to be largely an artefact of the models of word representation used; (ii) the proposed negative correlation between meaning change and prototypicality is shown to be much weaker than what has been claimed in prior art; and (iii) the proposed positive correlation between meaning change and polysemy is largely an artefact of word frequency. These empirical observations are corroborated by analytical proofs that show that count representations introduce an inherent dependence on word frequency, and thus word frequency cannot be evaluated as an independent factor with these representations.
