Datasets that boosted state-of-the-art solutions for Question Answering (QA) systems prove that it is possible to ask questions in natural language manner. However, users are still used to query-like systems where they type in keywords to search for answer. In this study we validate which parts of questions are essential for obtaining valid answer. In order to conclude that, we take advantage of LIME - a framework that explains prediction by local approximation. We find that grammar and natural language is disregarded by QA. State-of-the-art model can answer properly even if 'asked' only with a few words with high coefficients calculated with LIME. According to our knowledge, it is the first time that QA model is being explained by LIME.
