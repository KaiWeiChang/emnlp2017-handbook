Discourse deixis is a linguistic phenomenon in which pronouns have verbal or clausal, rather than nominal, antecedents. Studies have estimated that between 5\% and 10\% of pronouns in non-conversational data are discourse deictic. However, current coreference resolution systems ignore this phenomenon. This paper presents an automatic system for the detection and resolution of discourse-deictic pronouns. We introduce a two-step approach that first recognizes instances of discourse-deictic pronouns, and then resolves them to their verbal antecedent. Both components rely on linguistically motivated features. We evaluate the components in isolation and in combination with two state-of-the-art coreference resolvers. Results show that our system outperforms several baselines, including the only comparable discourse deixis system, and leads to a small but statistically significant improvement over the state-of-the-art in full coreference resolution. A post-hoc error analysis reveals the need for a less strict evaluation of this task.
