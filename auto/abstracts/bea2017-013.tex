Public speakings play important roles in schools and work places and properly using humor contributes to effective presentations. For the purpose of automatically evaluating speakers' humor usage, we build a presentation corpus containing humorous utterances based on TED talks. Compared to previous data resources supporting humor recognition research, ours has several advantages, including (a) both positive and negative instances coming from a homogeneous data set, (b) containing a large number of speakers, and (c) being open. Focusing on using lexical cues for humor recognition, we systematically compare a newly emerging text classification method based on Convolutional Neural Networks (CNNs) with a well-established conventional method using linguistic knowledge. The advantages of the CNN method are both getting higher detection accuracies and being able to learn essential features automatically.
