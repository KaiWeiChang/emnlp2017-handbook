We explore the hypothesis that emotion is one of the dimensions of language that surfaces from the native language into a second language. To check the role of emotions in native language identification (NLI), we model emotion information through polarity and emotion load features, and use document representations using these features to classify the native language of the author. The results indicate that emotion is relevant for NLI, even for high proficiency levels and across topics.
