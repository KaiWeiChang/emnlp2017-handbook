NLP applications for learners often rely on annotated learner corpora. Thereby, it is important that the annotations are both meaningful for the task, and consistent and reliable. We present a new longitudinal L1 learner corpus for German (handwritten texts collected in grade 2--4), which is transcribed and annotated with a target hypothesis that strictly only corrects orthographic errors, and is thereby tailored to research and tool development for orthographic issues in primary school. While for most corpora, transcription and target hypothesis are not evaluated, we conducted a detailed inter-annotator agreement study for both tasks. Although we achieved high agreement, our discussion of cases of disagreement shows that even with detailed guidelines, annotators differ here and there for different reasons, which should also be considered when working with transcriptions and target hypotheses of other corpora, especially if no explicit guidelines for their construction are known.
