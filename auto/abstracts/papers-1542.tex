An effective method to improve neural machine translation with monolingual data is to augment the parallel training corpus with back-translations of target language sentences. This work broadens the understanding of back-translation and investigates a number of methods to generate synthetic source sentences. We find that in all but resource poor settings back-translations obtained via sampling or noised beam outputs are most effective. Our analysis shows that sampling or noisy synthetic data gives a much stronger training signal than data generated by beam or greedy search. We also compare how synthetic data compares to genuine bitext and study various domain effects. Finally, we scale to hundreds of millions of monolingual sentences and achieve a new state of the art of 35 BLEU on the WMT'14 English-German test set.
