We present a corpus that encompasses the complete history of conversations between contributors to Wikipedia, one of the largest online collaborative communities. By recording the intermediate states of conversations - including not only comments and replies, but also their modifications, deletions and restorations - this data offers an unprecedented view of online conversation. Our framework is designed to be language agnostic, and we show that it extracts high quality data in both Chinese and English. This level of detail supports new research questions pertaining to the process (and challenges) of large-scale online collaboration. We illustrate the corpus' potential with two case studies on English Wikipedia that highlight new perspectives on earlier work. First, we explore how a person's conversational behavior depends on how they relate to the discussion's venue. Second, we show that community moderation of toxic behavior happens at a higher rate than previously estimated.
