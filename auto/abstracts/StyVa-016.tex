Sociolinguistic research suggests that speakers modulate their language style in response to their audience. Similar effects have recently been claimed to occur in the informal written context of Twitter, with users choosing less region-specific and non-standard vocabulary when addressing larger audiences. However, these studies have not carefully controlled for the possible confound of topic: that is, tweets addressed to a broad audience might also tend towards topics that engender a more formal style. In addition, it is not clear to what extent previous results generalize to different samples of users. Using mixed-effects models, we show that audience and topic have independent effects on the rate of distinctively Scottish usage in two demographically distinct Twitter user samples. However, not all effects are consistent between the two groups, underscoring the importance of replicating studies on distinct user samples before drawing strong conclusions from social media data.
