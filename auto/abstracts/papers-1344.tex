We challenge a common assumption in active learning, that a list-based interface populated by informative samples provides for efficient and effective data annotation. We show how a 2D scatterplot populated with diverse and representative samples can yield improved models given the same time budget. We consider this for bootstrapping-based information extraction, in particular named entity classification, where human and machine jointly label data. To enable effective data annotation in a scatterplot, we have developed an embedding-based bootstrapping model that learns the distributional similarity of entities through the patterns that match them in a large data corpus, while being discriminative with respect to human-labeled and machine-promoted entities. We conducted a user study to assess the effectiveness of these different interfaces, and analyze bootstrapping performance in terms of human labeling accuracy, label quantity, and labeling consensus across multiple users. Our results suggest that supervision acquired from the scatterplot interface, despite being noisier, yields improvements in classification performance compared with the list interface, due to a larger quantity of supervision acquired.
