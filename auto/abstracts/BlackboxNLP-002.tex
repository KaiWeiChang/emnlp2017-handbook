This paper addresses a relatively new task: prediction of  ASR performance on unseen broadcast programs. In a previous paper, we presented an ASR performance prediction system using CNNs that encode both text (ASR transcript) and speech, in order to predict word error rate. This work is  dedicated to the analysis of speech signal embeddings and text embeddings learnt by the CNN while training  our prediction model. We try to better understand which information is captured by the deep model and its relation with different conditioning factors. It is shown that hidden layers convey a clear signal about speech style, accent and broadcast type. We then try to leverage these 3 types of information at training time through multi-task learning. Our experiments show that this allows to train slightly more efficient ASR performance prediction systems that - in addition - simultaneously tag the analyzed utterances according to their speech style, accent and broadcast program origin.
