In many societies alcohol is a legal and common recreational substance and socially accepted. Alcohol consumption often comes along with social events as it helps people to increase their sociability and to overcome their inhibitions. On the other hand we know that increased alcohol consumption can lead to serious health issues, such as cancer, cardiovascular diseases and diseases of the digestive system, to mention a few. This work examines alcohol consumption during the FIFA Football World Cup 2018, particularly the usage of alcohol related information on Twitter. For this we analyse the tweeting behaviour and show that the tournament strongly increases the interest in beer. Furthermore we show that countries who had to leave the tournament at early stage might have done something good to their fans as the interest in beer decreased again.
