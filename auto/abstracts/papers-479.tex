Vlogs provide a rich public source of data in a novel setting. This paper examined the continuous sentiment styles employed in 27,333 vlogs using a dynamic intra-textual approach to sentiment analysis. Using unsupervised clustering, we identified seven distinct continuous sentiment trajectories characterized by fluctuations of sentiment throughout a vlog's narrative time. We provide a taxonomy of these seven continuous sentiment styles and found that vlogs whose sentiment builds up towards a positive ending are the most prevalent in our sample. Gender was associated with preferences for different continuous sentiment trajectories. This paper discusses the findings with respect to previous work and concludes with an outlook towards possible uses of the corpus, method and findings of this paper for related areas of research.
