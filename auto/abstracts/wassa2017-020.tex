There is a rich variety of data sets for sentiment analysis (viz.,{\textasciitilde}polarity and subjectivity classification). For the more challenging task of detecting discrete emotions following the definitions of Ekman and Plutchik, however, there are much fewer data sets, and notably no resources for the social media domain. This paper contributes to closing this gap by extending the \textit{SemEval 2016 stance and sentiment dataset} with emotion annotation. We (a) analyse annotation reliability and annotation merging; (b) investigate the relation between emotion annotation and the other annotation layers (stance, sentiment); (c) report modelling results as a baseline for future work.
