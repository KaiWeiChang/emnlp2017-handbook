We present a system for the detection of the stance of headlines with regard to their corresponding article bodies. The approach can be applied in fake news, especially clickbait detection scenarios. The component is part of a larger platform for the curation of digital content; we consider veracity and relevancy an increasingly important part of curating online information. We want to contribute to the debate on how to deal with fake news and related online phenomena with technological means, by providing means to separate related from unrelated headlines and further classifying the related headlines. On a publicly available data set annotated for the stance of headlines with regard to their corresponding article bodies, we achieve a (weighted) accuracy score of 89.59.
