The ``uniform information density'' (UID) hypothesis proposes that language producers aim for a constant rate of information flow within a message, and research on monologue-like written texts has found evidence for UID in production.  We consider conversational messages, using a large corpus of tweets, and look for UID behavior.  We do not find evidence of UID behavior, and even find context effects that are opposite that of previous, monologue-based research.  We propose that a more collaborative conception of information density and careful consideration of channel noise may be needed in the information-theoretic framework for conversation.
