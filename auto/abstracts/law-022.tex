Generics are linguistic expressions that make statements about or refer to kinds, or that report regularities of events. Non-generic expressions make statements about particular individuals or specific episodes. Generics are treated extensively in semantic theory (Krifka et al., 1995). In practice, it is often hard to decide whether a referring expression is generic or non-generic, and to date there is no data set which is both large and satisfactorily annotated. Such a data set would be valuable for creating automatic systems for identifying generic expressions, in turn facilitating knowledge extraction from natural language text. In this paper we provide the next steps for such an annotation endeavor. Our contributions are: (1) we survey the most important previous projects annotating genericity, focusing on resources for English; (2) with a new agreement study we identify problems in the annotation scheme of the largest currently available resource (ACE-2005); and (3) we introduce a linguistically-motivated annotation scheme for marking both clauses and their subjects with regard to their genericity. (4) We present a corpus of MASC (Ide et al., 2010) and Wikipedia texts annotated according to our scheme, achieving substantial agreement.
