Abstract: Rapid social change that some individuals perceive as antagonistic to their personal values or life expectation and thus experience as foisted upon them may elicit a whole spectrum of very different reactions and coping strategies --- from resigned adaption to actively engaging with the process and trying to counter its impact (Pinquart and Silbereisen 2004). The vast majority of individuals will however refrain from physical violence and choose other forms of engagement such as political protesting, voicing disconcert with current affairs or discussing them publicly, for example on social media platforms. Especially in discussions on non-partisan platforms, public discussions often derail because individuals tend to be highly emotionally engaged with social changes that impact their daily life and thus resort to various forms of interpersonal verbal attacks or targeting outgroups with ethnic slurs (ethnophaulisms). Intergroup swearing or ethnophaulisms in online discussions can thus be seen as another expression of low-intensity intergroup conflict. In this study, we study the extent, intensity and change in online swearing in the comment sections of YouTube videos from four popular German political talk shows before, during and after the height of the European refugee crisis. The talk show setting was chosen because a comparatively large segment of the German populations regularly watches them (i.e., around 15 percent market share), they cover a broad range of current topics ranging from fairly unrelated content regarding ethnic intergroup conflict such as to ``sugar as a drug'' to straight-up discussions of the refugee crisis and its consequences in Germany or Europe at large, the try tend to be non-partisan and spur a lot of discussion in the days after airing. Episodes of the talk shows are uploaded to YouTube shortly after airing on television and provide users with the opportunity to comment and engage in discussions. Here, we are interested in the process of social contagion whereby some behavior spreads through a network of users (Christakis and Fowler 2013), specifically when prior commenting behavior of actors in online forums creates conditions conducive or suppressive of voicing highly derogatory opinions towards others generally and minorities specifically. Our first research questions is thus: do preceding social media comments containing swearing target at ethnic minorities lower the threshold for other comments to subsequent target ethnic minorities using highly offensive language? In addition, we contend that intergroup conflict - whether interpersonal in the real-world or through online comments - is shaped by and responsive to impactful historic incidents, especially when those incidents are related to the core dimension of the intergroup conflict (Czymara and Schmidt-Catran 2017). In our application, three events stand out particularly: the sexual assaults during New Year's Eve 2015 in Cologne and two terrorist attacks: the axe attack in a regional train in Würzburg (July 18, 2016) and the Christmas market attack in Berlin (December 19, 2016). These incidents are at the heart of the German debates surrounding an appropriate response to the large influx of culturally dissimilar refugees and should elicit strong reactions from all parts of the discussion spectrum. The potential increase in offensive comments targeted towards ethnic minorities should through the process of social contagion in turn lower the threshold for other users to verbally attack minorities. Our second research questions thus asks whether ethnophaulisms  spread more rapidly in online comments in the aftermath of important incidents such as influential terrorist attacks. In total, we collected data from 96 videos, covering a total of 5,152 comments posted between June 2015 and May 2017. Our units of analysis are comment-reply threads where a user posts a comment and others react to it (more specifically, we have the following three-level hierarchical data structure: comments nested in parent comments nested in videos). For each comment, we measure our outcome by recording whether it contains interpersonal swearing or ethnophaulisms using two custom dictionaries. We analyze the relationship between swearing in preceding comments and actor's swearing behavior as well as how this relationship changes during periods of influential incidents using multilevel modeling techniques (see Figure 1). Because commenting in online social networks can be highly selective, we also construct user panels to track their commenting behavior over time. Results of our multilevel models for repeated cross-sectional data (Fairbrother 2013) suggest the presence of substantial contagion effects (see Table 1): we find that the probability of using ethnophaulisms increasing when a) the parent comment uses ethnophaulisms (+7 percentage points, pp) as well as b) the immediately preceding comment (+8pp). Moreover, we do not find that incidents of ethnophaulisms increase during periods of major incidents on average, but we do find strong support for the idea that incidents work as multipliers of social contagion: the effect of preceding comments using ethnophaulisms more than doubles (+18pp) during state changes from no incident to a major incident occurring. In order to explore the impact of selection effects, we also rely on user fixed effects regressions for users who commented on more than one video (n=251; Table 2). In essence, we observe very similar patterns to those of the multilevel models relying on the whole sample. However, the strength of association is considerably toned down suggesting that social contagion (+4pp) and multiplier effects (+2pp) are nonetheless at work but driven to considerable degree of self-selection of individuals.
