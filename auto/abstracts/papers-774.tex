This paper presents a simple framework for characterizing morphological complexity and how it encodes syntactic information. In particular, we propose a new measure of morpho-syntactic complexity in terms of governor-dependent preferential attachment that explains parsing performance. Through experiments on dependency parsing with data from Universal Dependencies (UD), we show that representations derived from morphological attributes deliver important parsing performance improvements over standard word form embeddings when trained on the same datasets. We also show that the new morpho-syntactic complexity measure is predictive of the gains provided by using morphological attributes over plain forms on parsing scores, making it a tool to distinguish languages using morphology as a syntactic marker from others.
