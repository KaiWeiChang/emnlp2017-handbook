In conjunction with this year's LAW theme, ``Syntactic Annotation of Non-canonical Language'' (NCL), I have been asked to weigh in on several important questions faced by anyone wishing to create annotated resources of NCLs. My experience with syntactic annotation of non-canonical language falls under an effort undertaken at Carnegie Mellon University with the aim of building an NLP pipeline for syntactic analysis of Twitter text. We designed a linguistically-grounded annotation scheme, applied it to tweets, and then trained statistical analyzers—first for part-of-speech (POS) tags (Gimpel et al., 2011; Owoputi et al., 2012), then for parses (Schneider et al., 2013; Kong et al., 2014). I will review some of the salient points from this work in addressing the broader questions about annotation methodology.
