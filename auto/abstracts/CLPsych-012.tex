Analyzing symptoms of schizophrenia, particularly on a large scale, has traditionally been challenging given the low prevalence of the condition: Around 1\% of the U.S. population suffers from this illness.  In this paper, we demonstrate that the social media data available on Twitter provides a platform to collect a large enough sample of self-identified schizophrenia sufferers, along with the language they use, to explore potential linguistic markers of schizophrenia. We describe several natural language processing (NLP) methods to analyze the language of schizophrenia as compared to the language of matched community controls, and examine how these signals interact with the widely-used LIWC categories for understanding mental health (Pennebaker et al., 2007).  We demonstrate a machine learning approach that can identify schizophrenia sufferers on Twitter in our balanced dataset with 82.3\% accuracy, and further, we find that our automatic NLP methods provide evidence of additional linguistic signals that may aid in identifying and getting help to people suffering from schizophrenia.
