The task of sentiment modification requires reversing the sentiment of the input and preserving the sentiment-independent content. However, aligned sentences with the same content but different sentiments are usually unavailable. Due to the lack of such parallel data, it is hard to extract sentiment independent content and reverse the sentiment in an unsupervised way. Previous work usually can not reconcile sentiment transformation and content preservation. In this paper, motivated by the fact the non-emotional context (e.g., ``staff'') provides strong cues for the occurrence of emotional words (e.g., ``friendly''), we propose a novel method that automatically extracts appropriate sentiment information from learned sentiment memories according to the specific context. Experiments show that our method substantially improves the content preservation degree and achieves the state-of-the-art performance.
