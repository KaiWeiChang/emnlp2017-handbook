Question difficulty estimates guide test creation, but are too costly for small-scale testing. We empirically verify that Bloom's Taxonomy, a standard tool for difficulty estimation during question creation, reliably predicts question difficulty observed after testing in a short-answer corpus. We also find that difficulty is mirrored in the amount of variation in student answers, which can be computed before grading. We show that question difficulty and its approximations are useful for \textit{automated grading}, allowing us to identify the optimal feature set for grading each question even in an unseen-question setting.
