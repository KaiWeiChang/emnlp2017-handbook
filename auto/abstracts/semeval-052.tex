This paper reports our submission to task 10 (Sentiment Analysis on Tweet, SAT) (Rosen thal et al., 2015) in SemEval 2015 , which contains five subtasks, i.e., contextual polarity disambiguation (subtask A: expression level), message polarity classification (subtask B: message-level), topic-based message polarity classification and detecting trends towards a topic (subtask C and D: topic-level), and determining sentiment strength of twitter terms (subtask E: term-level). For the first four subtasks, we built supervised models using traditional features and word embedding features to perform sentiment polarity classification. For subtask E, we first expanded the training data with the aid of external sentiment lexicons and then built a regression model to estimate the sentiment strength. Despite the sim- plicity of features, our systems rank above the average.
