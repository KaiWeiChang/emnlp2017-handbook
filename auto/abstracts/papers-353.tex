Character n-grams have been identified as the most successful feature in both single-domain and cross-domain Authorship Attribution (AA), but the reasons for their discriminative value were not fully understood. We identify subgroups of character n-grams that correspond to linguistic aspects commonly claimed to be covered by these features: morpho-syntax, thematic content and style. We evaluate the predictiveness of each of these groups in two AA settings: a single domain setting and a cross-domain setting where multiple topics are present. We demonstrate that character \$n\$-grams that capture information about affixes and punctuation account for almost all of the power of character n-grams as features. Our study contributes new insights into the use of n-grams for future AA work and other classification tasks.
