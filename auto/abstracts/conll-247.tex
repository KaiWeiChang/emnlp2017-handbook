We propose a post-OCR text correction approach for digitising texts in Romanised Sanskrit. Owing to the lack of resources our approach uses OCR models trained for other languages written in Roman. Currently, there exists no dataset available for Romanised Sanskrit OCR. So, we bootstrap a dataset of 430 images, scanned in two different settings and their corresponding ground truth. For training, we synthetically generate training images for both the settings. We find that the use of copying mechanism (Gu et al., 2016) yields a percentage increase of 7.69 in Character Recognition Rate (CRR) than the current state of the art model in solving monotone sequence-to-sequence tasks (Schnober et al., 2016). We find that our system is robust in combating OCR-prone errors, as it obtains a CRR of 87.01\% from an OCR output with CRR of 35.76\% for one of the dataset settings. A human judgement survey performed on the models shows that our proposed model results in predictions which are faster to comprehend and faster to improve for a human than the other systems.
