Sarcasm is a form of verbal irony that is intended to express contempt or ridicule. Often quoted as a challenge to sentiment analysis, sarcasm involves use of words of positive or no polarity to convey negative sentiment. Incongruity has been observed to be at the heart of sarcasm understanding in humans. Our work in sarcasm detection identifies different forms of incongruity and employs different machine learning techniques to capture them. This talk will describe the approach, datasets and challenges in sarcasm detection using different forms of incongruity. We identify two forms of incongruity: incongruity which can be understood based on the target text and common background knowledge, and incongruity which can be understood based on the target text and additional, specific context. The former involves use of sentiment-based features, word embeddings, and topic models. The latter involves creation of author's historical context based on their historical data, and creation of conversational context for sarcasm detection of dialogue.
