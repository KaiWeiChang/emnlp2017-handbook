In an e-commerce environment, user-oriented question-answering (QA) text pair could carry rich sentiment information. In this study, we propose a novel task/method to address QA sentiment analysis. In particular, we create a high-quality annotated corpus with specially-designed annotation guidelines for QA-style sentiment classification. On the basis, we propose a three-stage hierarchical matching network to explore deep sentiment information in a QA text pair. First, we segment both the question and answer text into sentences and construct a number of [Q-sentence, A-sentence] units in each QA text pair. Then, by leveraging a QA bidirectional matching layer, the proposed approach can learn the matching vectors of each [Q-sentence, A-sentence] unit. Finally, we characterize the importance of the generated matching vectors via a self-matching attention layer. Experimental results, comparing with a number of state-of-the-art baselines, demonstrate the impressive effectiveness of the proposed approach for QA-style sentiment classification.
