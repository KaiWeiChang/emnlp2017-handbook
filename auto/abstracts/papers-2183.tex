Neural NLP systems achieve high scores in the presence of sizable training dataset. Lack of such datasets leads to poor system performances in the case low-resource languages. We present two simple text augmentation techniques using dependency trees, inspired from image processing. We ``crop'' sentences by removing dependency links, and we ``rotate'' sentences by moving the tree fragments around the root. We apply these techniques to augment the training sets of low-resource languages in Universal Dependencies project. We implement a character-level sequence tagging model and evaluate the augmented datasets on part-of-speech tagging task. We show that crop and rotate provides improvements over the models trained with non-augmented data for majority of the languages, especially for languages with rich case marking systems.
