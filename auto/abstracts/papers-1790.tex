Prepositions are highly polysemous, and their variegated senses encode significant semantic information. In this paper we match each preposition's left- and right context,  and their interplay to the geometry of the word vectors to the left and right of the preposition. Extracting these features from a large corpus and using them with machine learning models makes for an efficient preposition sense disambiguation (PSD) algorithm, which is comparable to and better than state-of-the-art on two benchmark datasets. Our reliance on no  linguistic tool allows us to scale the PSD  algorithm  to a large corpus and learn sense-specific preposition representations. The crucial abstraction of preposition senses as word representations permits their use in downstream applications--phrasal verb paraphrasing and preposition selection--with new state-of-the-art results.
