We explore the use of segments learnt using Byte Pair Encoding (referred to as BPE units) as basic units for statistical machine translation between related languages and compare it with orthographic syllables, which are currently the best performing basic units for this translation task. BPE identifies the most frequent character sequences as basic units, while orthographic syllables are linguistically motivated pseudo-syllables. We show that BPE units modestly outperform orthographic syllables as units of translation, showing up to 11\% increase in BLEU score. While orthographic syllables can be used only for languages whose writing systems use vowel representations, BPE is writing system independent and we show that BPE outperforms other units for non-vowel writing systems too. Our results are supported by extensive experimentation spanning multiple language families and writing systems.
