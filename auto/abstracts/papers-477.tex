Neural machine translation represents an exciting leap forward in translation quality. But what longstanding weaknesses does it resolve, and which remain? We address these questions with a challenge set approach to translation evaluation and error analysis. A challenge set consists of a small set of sentences, each hand-designed to probe a system's capacity to bridge a particular structural divergence between languages.  To exemplify this approach, we present an English-French challenge set, and use it to analyze phrase-based and neural systems. The resulting analysis provides not only a more fine-grained picture of the strengths of neural systems, but also insight into which linguistic phenomena remain out of reach.
