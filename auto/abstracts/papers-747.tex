Websites' and mobile apps' privacy policies, written in natural language, tend to be long and difficult to understand. Information privacy revolves around the fundamental principle of Notice and choice, namely the idea that users should be able to make informed decisions about what information about them can be collected and how it can be used. Internet users want control over their privacy, but their choices are often hidden in long and convoluted privacy policy texts. Moreover, little (if any) prior work has been done to detect the provision of choices in text. We address this challenge of enabling user choice by automatically identifying and extracting pertinent choice language in privacy policies. In particular, we present a two-stage architecture of classification models to identify opt-out choices in privacy policy text, labelling common varieties of choices with a mean F1 score of 0.735. Our techniques enable the creation of systems to help Internet users to learn about their choices, thereby effectuating notice and choice and improving Internet privacy.
