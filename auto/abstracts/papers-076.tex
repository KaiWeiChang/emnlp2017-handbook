GRASP (GReedy Augmented Sequential Patterns) is an algorithm for automatically extracting patterns that characterize subtle linguistic phenomena. To that end, GRASP augments each term of input text with multiple layers of linguistic information. These different facets of the text terms are systematically combined to reveal rich patterns. We report highly promising experimental results in several challenging text analysis tasks within the field of Argumentation Mining. We believe that GRASP is general enough to be useful for other domains too. For example, each of the following sentences includes a claim for a [topic]: 1. Opponents often argue that the open primary is unconstitutional. [Open Primaries] 2. Prof. Smith suggested that affirmative action devalues the accomplishments of the chosen. [Affirmative Action] 3. The majority stated that the First Amendment does not guarantee the right to offend others. [Freedom of Speech] These sentences share almost no words in common, however, they are similar at a more abstract level. A human observer may notice the following underlying common structure, or pattern: [someone][argue/suggest/state][that][topic term][sentiment term]. GRASP aims to automatically capture such underlying structures of the given data. For the above examples it finds the pattern [noun][express][that][noun,topic][sentiment], where [express] stands for all its (in)direct hyponyms, and [noun,topic] means a noun which is also related to the topic.
