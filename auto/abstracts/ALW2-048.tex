This paper brings together theories from sociolinguistics and linguistic anthropology to critically evaluate the so-called ``language ideologies'' — the set of beliefs and ways of speaking about language — in the practices of abusive language classification in modern machine learning-based NLP. This argument is made at both a conceptual and empirical level, as we review approaches to abusive language from different fields, and use two neural network methods to analyze three datasets developed for abusive language classification tasks (drawn from Wikipedia, Facebook, and StackOverflow). By evaluating and comparing these results, we argue for the importance of incorporating theories of pragmatics and metapragmatics into both the design of classification tasks as well as in ML architectures.
