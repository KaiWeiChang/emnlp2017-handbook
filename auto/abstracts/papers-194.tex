Monolingual evaluation of Machine Translation (MT) aims to simplify human assessment by requiring assessors to compare the meaning of the MT output with a reference translation, opening up the task to a much larger pool of genuinely qualified evaluators. Monolingual evaluation runs the risk, however, of bias in favour of MT systems that happen to produce translations superficially similar to the reference and, consistent with this intuition, previous investigations have concluded monolingual assessment to be strongly biased in this respect. On re-examination of past analyses, we identify a series of potential analytical errors that force some important questions to be raised about the reliability of past conclusions, however. We subsequently carry out further investigation into reference bias via direct human assessment of MT adequacy via quality controlled crowd-sourcing. Contrary to both intuition and past conclusions, results for show no significant evidence of reference bias in monolingual evaluation of MT.
