Lexical Simplification is the task of modifying the lexical content of complex sentences in order to make them simpler. Due to the lack of reliable resources available for the task, most existing approaches have difficulties producing simplifications which are grammatical and that preserve the meaning of the original text. In order to improve on the state-of-the-art of this task, we propose user studies with non-native speakers, which will result in new, sizeable datasets, as well as novel ways of performing Lexical Simplification. The results of our first experiments show that new types of classifiers, along with the use of additional resources such as spoken text language models, produce the state-of-the-art results for the Lexical Simplification task of SemEval-2012.
