Homographic puns have a long history in human writing, widely used in written and spoken literature, which usually occur in a certain syntactic or stylistic structure. How to recognize homographic puns is an important research. However, homographic pun recognition does not solve very well in existing work. In this work, we first use WordNet to understand and expand word embedding for settling the polysemy of homographic puns, and then propose a WordNet-Encoded Collocation-Attention network model (WECA) which combined with the context weights for recognizing the puns. Our experiments on the SemEval2017 Task7 and Pun of the Day demonstrate that the proposed model is able to distinguish between homographic pun and non-homographic pun texts. We show the effectiveness of the model to present the capability of choosing qualitatively informative words. The results show that our model achieves the state-of-the-art performance on homographic puns recognition.
