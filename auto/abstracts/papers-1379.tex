To what extent could the sommelier profession, or wine stewardship, be displaced by machine leaning algorithms? There are at least three essential skills that make a qualified sommelier: wine theory, blind tasting, and beverage service, as exemplified in the rigorous certification processes of certified sommeliers and above (advanced and master) with the most authoritative body in the industry, the Court of Master Sommelier (hereafter CMS). We propose and train corresponding machine learning models that match these skills, and compare algorithmic results with real data collected from a large group of wine professionals. We find that our machine learning models outperform human sommeliers on most tasks — most notably in the section of blind tasting, where hierarchically supervised Latent Dirichlet Allocation outperforms sommeliers' judgment calls by over 6\% in terms of F1-score; and in the section of beverage service, especially wine and food pairing, a modified Siamese neural network based on BiLSTM achieves better results than sommeliers by 2\%. This demonstrates, contrary to popular opinion in the industry, that the sommelier profession is at least to some extent automatable, barring economic (Kleinberg et al., 2017) and psychological (Dietvorst et al., 2015) complications.
