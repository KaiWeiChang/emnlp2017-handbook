Detection and correction of errors and inconsistencies in ``gold treebanks'' are becoming more and more central topics of corpus annotation. The paper illustrates a new incremental method for enhancing treebanks, with particular emphasis on the extension of error patterns across different textual genres and registers. Impact and role of corrections have been assessed in a dependency parsing experiment carried out with four different parsers, whose results are promising. For both evaluation datasets, the performance of parsers increases, in terms of the standard LAS and UAS measures and of a more focused measure taking into account only relations involved in error patterns, as well as at the level of individual dependencies.
