We present supertagging-based models for Tree Adjoining Grammar parsing that use neural network architectures and dense vector representation of supertags (elementary trees) to achieve state-of-the-art performance in unlabeled and labeled attachment scores. The shift-reduce parsing model eschews lexical information entirely, and uses only the 1-best supertags to parse a sentence, providing further support for the claim that supertagging is ``almost parsing.'' We demonstrate that the embedding vector representations the parser induces for supertags possess linguistically interpretable structure, supporting analogies between grammatical structures like those familiar from recent work in distributional semantics. This dense representation of supertags overcomes the drawbacks for statistical models of TAG as compared to CCG parsing, raising the possibility that TAG is a viable alternative for NLP tasks that require the assignment of richer structural descriptions to sentences.
