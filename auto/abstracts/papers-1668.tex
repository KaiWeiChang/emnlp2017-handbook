When the semantics of a sentence are not representable in a semantic parser's output schema, parsing will inevitably fail. Detection of these instances is commonly treated as an out-of-domain classification problem. However, there is also a more subtle scenario in which the test data is drawn from the same domain. In addition to formalizing this problem of domain-adjacency, we present a comparison of various baselines that could be used to solve it. We also propose a new simple sentence representation that emphasizes words which are unexpected. This approach improves the performance of a downstream semantic parser run on in-domain and domain-adjacent instances.
