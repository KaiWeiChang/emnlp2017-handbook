It is a challenging task to automatically compose poems with not only fluent expressions but also aesthetic wording. Although much attention has been paid to this task and promising progress is made, there exist notable gaps between automatically generated ones with those created by humans, especially on the aspects of term novelty and thematic consistency. Towards filling the gap, in this paper, we propose a conditional variational autoencoder with adversarial training for classical Chinese poem generation, where the autoencoder part generates poems with novel terms and a discriminator is applied to adversarially learn their thematic consistency with their titles. Experimental results on a large poetry corpus confirm the validity and effectiveness of our model, where its automatic and human evaluation scores outperform existing models.
