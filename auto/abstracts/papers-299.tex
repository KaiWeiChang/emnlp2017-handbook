Verifiability is one of the core editing principles in Wikipedia, where editors are encouraged to provide  citations for the added content. For a Wikipedia article determining what content is covered by a citation or the citation span is not trivial, an important aspect for automated citation finding for uncovered content, or fact assessments. We address the problem of determining the citation span in Wikipedia articles. We approach this problem by classifying which textual fragments in an article are covered or hold true given a citation. We propose a sequence classification approach where for a paragraph and a citation, we determine the citation span at a fine-grained level. We provide a thorough experimental evaluation and compare our approach against baselines adopted from the scientific domain, where we show improvement for all evaluation metrics.
