Explicit Semantic Analysis (ESA) utilizes the Wikipedia knowledge base to represent the semantics of a word by a vector where every dimension refers to an explicitly defined concept like a Wikipedia article. ESA inherently assumes that Wikipedia concepts are orthogonal to each other, therefore, it considers that two words are related only if they co-occur in the same articles. However, two words can be related to each other even if they appear separately in related articles rather than co-occurring in the same articles. This leads to a need for extending the ESA model to consider the relatedness between the explicit concepts (i.e. Wikipedia articles in Wikipedia based implementation) for computing textual relatedness. In this paper, we present Non-Orthogonal ESA (NESA) which represents more fine grained semantics of a word as a vector of explicit concept dimensions, where every such concept dimension further constitutes a semantic vector built in another vector space. Thus, NESA considers the concept correlations in computing the relatedness between two words. We explore different approaches to compute the concept correlation weights, and compare these approaches with other existing methods. Furthermore, we evaluate our model NESA on several word relatedness benchmarks showing that it outperforms the state of the art methods.
