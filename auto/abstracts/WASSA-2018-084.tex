Past shared tasks on emotions use data with both overt expressions of emotions (I am so happy to see you!) as well as subtle expressions where the emotions have to be inferred, for instance from event descriptions. Further, most datasets do not focus on the cause or the stimulus of the emotion. Here, for the first time, we propose a shared task where systems have to predict the emotions in a large automatically labeled dataset of tweets without access to words denoting emotions. Based on this intention, we call this the Implicit Emotion Shared Task (IEST) because the systems have to infer the emotion mostly from the context. Every tweet has an occurrence of an explicit emotion word that is masked. The tweets are collected in a manner such that they are likely to include a description of the cause of the emotion --- the stimulus. Altogether, 30 teams submitted results which range from macro F1 scores of 21 \% to 71 \%. The baseline (Max- Ent bag of words and bigrams) obtains an F1 score of 60 \% which was available to the participants during the development phase. A study with human annotators suggests that automatic methods outperform human predictions, possibly by honing into subtle textual clues not used by humans. Corpora, resources, and results are available at the shared task website at http://implicitemotions.wassa2018.com.
