Despite the significant improvement of data-driven dependency parsing systems in recent years, they still achieve a considerably lower performance in parsing spoken language data in comparison to written data. On the example of Spoken Slovenian Treebank, the first spoken data treebank using the UD annotation scheme, we investigate which speech-specific phenomena undermine parsing performance, through a series of training data and treebank modification experiments using two distinct state-of-the-art parsing systems. Our results show that segmentation is the most prominent cause of low parsing performance, both in parsing raw and pre-segmented transcriptions. In addition to shorter utterances, both parsers perform better on normalized transcriptions including basic markers of prosody and excluding disfluencies, discourse markers and fillers. On the other hand, the effects of written training data addition and speech-specific dependency representations largely depend on the parsing system selected.
