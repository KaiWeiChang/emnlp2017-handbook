In recent years, the natural language processing community has moved away from task-specific feature engineering, i.e., researchers discovering ad-hoc feature representations for various tasks, in favor of general-purpose methods that learn the input representation by themselves. However, state-of-the-art approaches to disfluency detection in spontaneous speech transcripts currently still depend on an array of hand-crafted features, and other representations derived from the output of pre-existing systems such as language models or dependency parsers. As an alternative, this paper proposes a simple yet effective model for automatic disfluency detection, called an auto-correlational neural network (ACNN). The model uses a convolutional neural network (CNN) and augments it with a new auto-correlation operator at the lowest layer that can capture the kinds of ``rough copy'' dependencies that are characteristic of repair disfluencies in speech. In experiments, the ACNN model outperforms the baseline CNN on a disfluency detection task with a 5\% increase in f-score, which is close to the previous best result on this task.
