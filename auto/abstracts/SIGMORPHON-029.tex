Modeling morphological inflection is an important task in Natural Language Processing. In contrast to earlier work that has largely used orthographic representations, we experiment with this task in a phonetic character space, representing inputs as either IPA segments or bundles of phonological distinctive features. We show that both of these inputs, somewhat counterintuitively, achieve similar accuracies on morphological inflection, slightly lower than orthographic models. We conclude that providing detailed phonological representations is largely redundant when compared to IPA segments, and that articulatory distinctions relevant for word inflection are already latently present in the distributional properties of many graphemic writing systems.
