In multi-sense word embeddings, contextual variations in corpus may cause a  univocal word to be embedded into different sense vectors. Shi et al.(2016) show that this kind of pseudo multi-senses can be eliminated by linear transformations. In this paper, we show that pseudo multi-senses may come from a uniform and meaningful phenomenon such as subjective and sentimental usage, though they are seemingly redundant. We present an unsupervised algorithm to find a linear transformation which can minimize the transformed distance of a group of sense pairs. The major shrinking direction of this transformation  is found to be related with subjective shift. Therefore, we can not only eliminate pseudo multi-senses in multi-sense embeddings, but also identify these subjective senses and tag the subjective and sentimental usage of words in the corpus automatically.
