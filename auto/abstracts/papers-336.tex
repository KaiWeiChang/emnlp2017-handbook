Representation learning is the dominant technique for unsupervised domain adaptation, but existing approaches have two major weaknesses. First, they often require the specification of ``pivot features'' that generalize across domains, which are selected by task-specific heuristics. We show that a novel but simple feature embedding approach provides better performance, by exploiting the feature template structure common in NLP problems. Second, unsupervised domain adaptation is typically treated as a task of moving from a single source to a single target domain. In reality, test data may be diverse, relating to the training data in some ways but not others. We propose an alternative formulation, in which each instance has a vector of domain attributes, can be used to learn distill the domain-invariant properties of each feature.
