Transfer learning (TL) proposes to enhance machine learning performance on a  problem, by reusing labeled data originally designed for a related problem. In particular, domain adaptation consists, for a specific task, in reusing training data developed for the same task but a distinct domain. This is particularly relevant to the applications of deep learning in Natural Language Processing, because those usually require large annotated corpora that may not exist for the targeted domain, but exist for side domains. In this paper, we experiment TL for the task of Relation Extraction (RE) from biomedical texts, using the TreeLSTM model. We empirically show the interest of TreeLSTM alone and with domain adaptation by obtaining better performances than the state of the art on two biomedical RE tasks and equal performances for two others, for which few annotated data are available. Furthermore, we propose an analysis of the role that syntactic features may play in TL for RE.
