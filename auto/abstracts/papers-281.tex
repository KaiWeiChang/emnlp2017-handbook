Previous work on dialog act (DA) classification has investigated different methods, such as hidden Markov models, maximum entropy, conditional random fields, graphical models, and support vector machines. A few recent studies explored using deep learning neural networks for DA classification, however, it is not clear yet what is the best method for using dialog context or DA sequential information, and how much gain it brings. This paper proposes several ways of using context information for DA classification, all in the deep learning framework. The baseline system classifies each utterance using the convolutional neural networks (CNN). Our proposed methods include using hierarchical models (recurrent neural networks (RNN) or CNN) for DA sequence tagging where the bottom layer takes the sentence CNN representation as input, concatenating predictions from the previous utterances with the CNN vector for classification, and performing sequence decoding based on the predictions from the sentence CNN model. We conduct thorough experiments and comparisons on the Switchboard corpus, demonstrate that incorporating context information significantly improves DA classification, and show that we achieve new state-of-the-art performance for this task.
