In this work, we review popular representation learning methods for the task of hate speech detection on Twitter data-. Methods in representation learning have been successfully applied to a variety of NLP tasks, including sentiment analysis \cite{tang2016sentiment}, sarcasm detection \cite{DBLP:conf/emnlp/JoshiTPBC16}, and text similarity \cite{DBLP:conf/acl/KenterBR16}. Specifically, for the task of hate speech detection, representation learning methods have shown promising results in comparison to traditional feature-based methods \cite{badjatiya2017deep,nobata2016abusive,djuric2015hate}. However, there has been no comprehensive study comparing the utility of embedding methods for hate speech detection. To fill this gap in literature, we study the effect of word and sentence embeddings methods for hate speech detection on publicly available data sets.
