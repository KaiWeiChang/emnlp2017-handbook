Grammatical error correction, like other machine learning tasks, greatly benefits from large quantities of high quality training data, which is typically expensive to produce. While writing a program to automatically generate realistic grammatical errors would be difficult, one could learn the distribution of naturally-occurring errors and attempt to introduce them into other datasets. Initial work on inducing errors in this way using statistical machine translation has shown promise; we investigate cheaply constructing synthetic samples, given a small corpus of human-annotated data, using an off-the-rack attentive sequence-to-sequence model and a straight-forward post-processing procedure. Our approach yields error-filled artificial data that helps a vanilla bi-directional LSTM to outperform the previous state of the art at grammatical error detection, and a previously introduced model to gain further improvements of over 5\% F0.5 score. When attempting to determine if a given sentence is synthetic, a human annotator at best achieves 39.39 F1 score, indicating that our model generates mostly human-like instances.
