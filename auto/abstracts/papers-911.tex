Many tasks in natural language processing involve comparing two sentences to compute some notion of relevance, entailment, or similarity.  Typically this comparison is done either at the word level or at the sentence level, with no attempt to leverage the inherent structure of the sentence.   When sentence structure is used for comparison, it is obtained during a non-differentiable pre-processing step, leading to propagation of errors.  We introduce a model of structured alignments between sentences, showing how to compare two sentences by matching their latent structures.  Using a structured attention mechanism, our model matches candidate spans in the first sentence to candidate spans in the second sentence, simultaneously discovering the tree structure of each sentence. Our model  is fully differentiable and trained only on the matching objective. We evaluate this model on two  tasks, natural  entailment detection and answer sentence selection, and find that modeling latent tree structures results in superior performance. Analysis of the  learned sentence structures  shows they can reflect some syntactic phenomena.
