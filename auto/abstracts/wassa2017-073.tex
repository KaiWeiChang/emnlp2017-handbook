Different theories posit different sources for feelings of well-being and happiness.  Appraisal theory grounds our emotional responses in our goals and desires and their fulfillment, or lack of fulfillment. Self-Determination theory posits that the basis for well-being rests on our assessments of our competence, autonomy and social connection. And surveys that measure happiness empirically note that people require their basic needs to be met for food and shelter, but beyond that tend to be happiest when socializing, eating or having sex. We analyze a corpus of private micro-blogs from a well-being application called Echo, where users label each written post about daily events with a happiness score between 1 and 9.  Our goal is to ground the linguistic descriptions of events that users experience in theories of well-being and happiness, and then examine the extent to which different theoretical accounts can explain the variance in the happiness scores.  We show that recurrent event types, such as obligation and incompetence, which affect people's feelings of well-being are not captured in current lexical or semantic resources.
