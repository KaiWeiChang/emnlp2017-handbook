User intent detection plays a critical role in question-answering and dialog systems. Most previous works treat intent detection as a classification problem where utterances are labeled with predefined intents. However, it is labor-intensive and time-consuming to label users' utterances as intents are diversely expressed and novel intents will continually be involved. Instead, we study the zero-shot intent detection problem, which aims to detect emerging user intents where no labeled utterances are currently available. We propose two capsule-based architectures: IntentCapsNet that extracts semantic features from utterances and aggregates them to discriminate existing intents, and IntentCapsNet-ZSL which gives IntentCapsNet the zero-shot learning ability to discriminate emerging intents via knowledge transfer from existing intents. Experiments on two real-world datasets show that our model not only can better discriminate diversely expressed existing intents, but is also able to discriminate emerging intents when no labeled utterances are available.
