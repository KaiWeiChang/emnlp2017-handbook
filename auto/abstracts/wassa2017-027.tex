There has been a good amount of progress in sentiment analysis over the past 10 years, including the proposal of new methods and the creation of benchmark datasets. In some papers, however, there is a tendency to compare models only on one or two datasets, either because of time restraints or because the model is tailored to a specific task. Accordingly, it is hard to understand how well a certain model generalizes across different tasks and datasets. In this paper, we contribute to this situation by comparing several models on six different benchmarks, which belong to different domains and additionally have different levels of granularity (binary, 3-class, 4-class and 5-class). We show that Bi-LSTMs perform well across datasets and that both LSTMs and Bi-LSTMs are particularly good at fine-grained sentiment tasks (\ie, with more than two classes). Incorporating sentiment information into word embeddings during training gives good results for datasets that are lexically similar to the training data. With our experiments, we contribute to a better understanding of the performance of different model architectures on different data sets. Consequently, we detect novel state-of-the-art results on the \textit{SenTube} datasets.
