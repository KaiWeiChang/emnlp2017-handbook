We investigate how encoder-decoder models trained on a synthetic dataset of task-oriented dialogues process disfluencies, such as hesitations and self-corrections. We find that, contrary to earlier results, disfluencies have very little impact on the task success of seq-to-seq models with attention. Using visualisations and diagnostic classifiers, we analyse the representations that are incrementally built by the model, and discover that models develop little to no awareness of the structure of disfluencies. However, adding disfluencies to the data appears to help the model create clearer representations overall, as evidenced by the attention patterns the different models exhibit.
