This paper introduces the Bank Question (BQ) corpus, a Chinese corpus for sentence semantic equivalence identification (SSEI). The BQ corpus contains 120,000 question pairs from 1-year online bank custom service logs. To efficiently process and annotate questions from such a large scale of logs, this paper proposes a clustering based annotation method to achieve questions with the same intent. First, the deduplicated questions with the same answer are clustered into stacks by the Word Mover's Distance (WMD) based Affinity Propagation (AP) algorithm. Then, the annotators are asked to assign the clustered questions into different intent categories. Finally, the positive and negative question pairs for SSEI are selected in the same intent category and between different intent categories respectively. We also present six SSEI benchmark performance on our corpus, including state-of-the-art algorithms. As the largest manually annotated public Chinese SSEI corpus in the bank domain, the BQ corpus is not only useful for Chinese question semantic matching research, but also a significant resource for cross-lingual and cross-domain SSEI research. The corpus is available in public.
