Simple reference games are of central theoretical and empirical importance in the study of situated language use.  Although language provides rich, compositional truth-conditional semantics to facilitate reference, speakers and listeners may sometimes lack the overall lexical and cognitive resources to guarantee successful reference through these means alone.  However, language also has rich associational structures that can serve as a further resource for achieving successful reference. Here we investigate this use of associational information in a setting where only associational information is available: a simplified version of the popular game Codenames. Using optimal experiment design techniques, we compare a range of models varying in the type of associative information deployed and in level of pragmatic sophistication against human behavior.  In this setting we find that listeners' behavior reflects direct bigram collocational associations more strongly than word-embedding or semantic knowledge graph-based associations and that there is little evidence for pragmatically sophisticated behavior on the part of either speakers or listeners. More generally, we demonstrate the effective use of simple tasks to derive insights into the nature of complex linguistic phenomena.
