Having an entity annotated corpus of the clinical domain is one of the basic requirements for detection of clinical entities using machine learning (ML) approaches. Past researches have shown the superiority of statistical/ML approaches over the rule based approaches. But in order to take full advantage of the ML approaches, an accurately annotated corpus becomes an essential requirement. Though there are a few annotated corpora available either on a small data set, or covering a narrower domain (like cancer patients records, lab reports), annotation of a large data set representing the entire clinical domain has not been created yet. In this paper, we have described in detail the annotation guidelines, annotation process and our approaches in creating a CER (clinical entity recognition) corpus of 5,160 clinical documents from forty different clinical specialities. The clinical entities range across various types such as diseases, procedures, medications, medical devices and so on. We have classified them into eleven categories for annotation. Our annotation also reflects the relations among the group of entities that constitute larger concepts altogether.
