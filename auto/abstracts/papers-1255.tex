Natural language constitutes a predominant medium for much of human learning and pedagogy. We consider the problem of concept learning from natural language explanations, and a small number of labeled examples of the concept. For example, in learning the concept of a phishing email, one might say `this is a phishing email because it asks for your bank account number'. Solving this problem involves both learning to interpret open ended natural language statements, and learning the concept itself. We present a joint model for (1) language interpretation (semantic parsing) and (2) concept learning (classification) that does not require labeling statements with logical forms. Instead, the model prefers discriminative interpretations of statements in context of observable features of the data as a weak signal for parsing. On a dataset of email-related concepts, our approach yields across-the-board improvements in classification performance, with a 30\% relative improvement in F1 score over competitive methods in the low data regime.
