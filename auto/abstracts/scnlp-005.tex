Silence is an integral part of the most frequent turn-taking phenomena in spoken conversations.  Silence is sized and placed within the conversation flow and it is coordinated by the speakers along with the other speech acts. The objective of this analytical study is twofold: to explore the functions of silence with duration of one second and above, towards information flow in a dyadic conversation utilizing the sequences of dialog acts present in the turns surrounding the silence itself; and to design a feature space useful for clustering the silences using a hierarchical concept formation algorithm. The resulting clusters are manually grouped into functional categories based on their similarities. It is observed that the silence plays an important role in response preparation, also can indicate speakers' hesitation or indecisiveness. It is also observed that sometimes long silences can be used deliberately to get a forced response from another speaker thus making silence a multi-functional and an important catalyst towards information flow.
