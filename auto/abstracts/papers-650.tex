We introduce a novel sub-character architecture that exploits a unique compositional structure of the Korean language. Our method decomposes each character into a small set of primitive phonetic units called jamo letters from which character- and word-level representations are induced. The jamo letters divulge syntactic and semantic information that is difficult to access with conventional character-level units. They greatly alleviate the data sparsity problem, reducing the observation space to 1.6\% of the original while increasing accuracy in our experiments. We apply our architecture to dependency parsing and achieve dramatic improvement over strong lexical baselines.
