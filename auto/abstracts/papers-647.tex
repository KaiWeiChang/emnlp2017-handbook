Supervised morphological paradigm learning by identifying and aligning the longest common subsequence found in inflection tables has recently been proposed as a simple yet competitive way to induce morphological patterns.  We combine this non-probabilistic strategy of inflection table generalization with a discriminative classifier to permit the reconstruction of complete inflection tables of unseen words.  Our system learns morphological paradigms from labeled examples of inflection patterns (inflection tables) and then produces inflection tables from unseen lemmas or base forms.  We evaluate the approach on datasets covering 11 different languages and show that this approach results in consistently higher accuracies vis-à-vis other methods on the same task, thus indicating that the general method is a viable approach to quickly creating high-accuracy morphological resources.
