In times of fake news and alternative facts, pro and con arguments on controversial topics are of increasing importance. Recently, we presented args.me as the first search engine for arguments on the web. In its initial version, args.me ranked arguments solely by their relevance to a topic queried for, making it hard to learn about the diverse topical aspects covered by the search results. To tackle this shortcoming, we integrated a visualization interface for result exploration in args.me that provides an instant overview of the main aspects in a barycentric coordinate system. This topic space is generated ad-hoc from controversial issues on Wikipedia and argument-specific LDA models. In two case studies, we demonstrate how individual arguments can be found easily through interactions with the visualization, such as highlighting and filtering.
