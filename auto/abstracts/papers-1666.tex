Maximum-likelihood estimation (MLE) is one of the most widely used approaches for training structured prediction models for text-generation based natural language processing applications. However, besides exposure bias, models trained with MLE suffer from wrong objective problem where they are trained to maximize the word-level correct next step prediction, but are evaluated with respect to sequence-level discrete metrics such as ROUGE and BLEU. Several variants of policy-gradient methods address some of these problems by optimizing for final discrete evaluation metrics and showing improvements over MLE training for downstream tasks like text summarization and machine translation. However, policy-gradient methods suffers from high sample variance, making the training process very difficult and unstable.  In this paper, we present an alternative direction towards mitigating this problem by introducing a new objective (CaLcs) based on a differentiable surrogate of longest common subsequence (LCS) measure that captures sequence-level structure similarity. Experimental results on abstractive summarization and machine translation validate the effectiveness of the proposed approach.
