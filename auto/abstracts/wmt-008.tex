The system described here deliberately avoids any statistical elements in translation process. The basic assumption is that running text can always be decomposed into structured units, and that these units can be described on more or less general level. The translation is not performed on the basis of surface word forms, but rather as a controlled sequence of operations, where the text in source language is processed into surface form of the target language. The basic components in the system are the lexicon and grammar of both languages. On the abstract level, the language can be described by means of tags, each of which represents various degrees of abstractness. For example POS tags are the most abstract ones, each representing a large set of members, whereas word lemmas are least abstract, and morphological tags are somewhere in between. The combination of the tags constitutes the knowledge, on the basis of which the text is converted into the surface form of the target language.
