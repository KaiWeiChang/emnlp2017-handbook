We propose a new sentence simplification task (Split-and-Rephrase) where the aim is to split a complex sentence into a meaning preserving sequence of shorter sentences. Like sentence simplification, splitting-and-rephrasing has the potential of benefiting both natural language processing and societal applications. Because shorter sentences are generally better processed by NLP systems, it could be used as a preprocessing step which facilitates and improves the performance of parsers, semantic role labellers and machine translation systems. It should also be of use for people with reading disabilities because it allows the conversion of longer sentences into shorter ones. This paper makes two contributions towards this new task. First, we create and make available a benchmark consisting of 1,066,115 tuples mapping a single complex sentence to a sequence of sentences expressing the same meaning. Second, we propose five models (vanilla sequence-to-sequence to semantically-motivated models) to understand the difficulty of the proposed task.
