Written sentences can be more ambiguous than spoken sentences. We investigate this difference for two different types of ambiguity: prepositional phrase (PP) attachment and sentences where the addition of commas changes the meaning. We recorded a native English speaker saying several of each type of sentence both with and without disambiguating contextual information.  These sentences were then presented either as text or audio and either with or without context to subjects who were asked to select the proper interpretation of the sentence. Results suggest that comma-ambiguous sentences are easier to disambiguate than PP-attachment-ambiguous sentences, possibly due to the presence of clear prosodic boundaries, namely silent pauses. Subject performance for sentences with PP-attachment ambiguity without context was 52\% for text only while it was 72.4\% for audio only, suggesting that audio has more disambiguating information than text. Using an analysis of acoustic features of two PP-attachment sentences, a simple classifier was implemented to resolve the PP-attachment ambiguity being early or late closure with a mean accuracy of 80\%.
