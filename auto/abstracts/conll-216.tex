Inducing sparseness while training neural networks has been shown to yield models with a lower memory footprint but similar effectiveness to dense models. However, sparseness is typically induced starting from a dense model, and thus this advantage does not hold during training. We propose techniques to enforce sparseness upfront in recurrent sequence models for NLP applications, to also benefit training. First, in language modeling, we show how to increase hidden state sizes in recurrent layers without increasing the number of parameters, leading to more expressive models. Second, for sequence labeling, we show that word embeddings with predefined sparseness lead to similar performance as dense embeddings, at a fraction of the number of trainable parameters.
