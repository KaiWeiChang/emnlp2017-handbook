Recently, string kernels have obtained state-of-the-art results in various text classification tasks such as Arabic dialect identification or native language identification. In this paper, we apply two simple yet effective transductive learning approaches to further improve the results of string kernels. The first approach is based on interpreting the pairwise string kernel similarities between samples in the training set and samples in the test set as features. Our second approach is a simple self-training method based on two learning iterations. In the first iteration, a classifier is trained on the training set and tested on the test set, as usual. In the second iteration, a number of test samples (to which the classifier associated higher confidence scores) are added to the training set for another round of training. However, the ground-truth labels of the added test samples are not necessary. Instead, we use the labels predicted by the classifier in the first training iteration. By adapting string kernels to the test set, we report significantly better accuracy rates in English polarity classification and Arabic dialect identification.
