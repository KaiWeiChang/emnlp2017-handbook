We present MoodSwipe, a soft keyboard that suggests text messages given the user-specified emotions utilizing the real dialog data. The aim of MoodSwipe is to create a convenient user interface to enjoy the technology of emotion classification and text suggestion, and at the same time to collect labeled data automatically for developing more advanced technologies. While users select the MoodSwipe keyboard, they can type as usual but sense the emotion conveyed by their text and receive suggestions for their message as a benefit. In MoodSwipe, the detected emotions serve as the medium for suggested texts, where viewing the latter is the incentive to correcting the former. We conduct several experiments to show the superiority of the emotion classification models trained on the dialog data, and further to verify good emotion cues are important context for text suggestion.
