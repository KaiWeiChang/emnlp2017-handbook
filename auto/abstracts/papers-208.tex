There is growing interest in the language developed by agents interacting in emergent-communication settings. Earlier studies have focused on the agents' symbol usage, rather than on their representation of visual input. In this paper, we consider the referential games of Lazaridou et al. (2017), and investigate the representations the agents develop during their evolving interaction. We find that the agents establish successful communication by inducing visual representations that almost perfectly align with each other, but, surprisingly, do not capture the conceptual properties of the objects depicted in the input images. We conclude that, if we care about developing language-like communication systems, we must pay more attention to the visual semantics agents associate to the symbols they use.
