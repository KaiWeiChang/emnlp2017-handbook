Emotions can be triggered by various factors. According to the Appraisal Theories (De Rivera, 1977; Frijda, 1986; Ortony et al., 1988; Johnson-Laird and Oatley, 1989) emotions are elicited and differentiated on the basis of the cognitive evaluation of the personal significance of a situation, object or event based on ``appraisal criteria'' (intrinsic characteristics of objects and events, significance of events to individual needs and goals, individual's ability to cope with the consequences of the event, compatibility of event with social or personal standards, norms and values). These differences in values can trigger reactions such as anger, disgust (contempt), sadness, etc., because these behaviors are evaluated by the public as being incompatible with their social/personal standards, norms or values. Such arguments are frequently present both in mainstream media, as well as social media, building a society-wide view, attitude and emotional reaction towards refugees/immigrants. In this demo, I will talk about experiments to annotate and detect factual arguments that are linked to human needs/motivations from text and in consequence trigger emotion in the media audience and propose a new task for next year's WASSA.
