Statistical Machine Translation has come a long way improving the translation quality of a range of different linguistic phenomena. With negation however, techniques proposed and implemented for improving translation performance on negation have simply followed from the developers' beliefs about why performance is worse. These beliefs, however, have never been validated by an error analysis of the translation output. In contrast, the current paper shows that an informative empirical error analysis can be formulated in terms of (1) the set of semantic elements involved in the meaning of negation, and (2) a small set of string-based operations that can characterise errors in the translation of those elements. Results on a Chinese-to-English translation task confirm the robustness of our analysis cross-linguistically and the basic assumptions can inform an automated investigation into the causes of translation errors. Conclusions drawn from this analysis should guide future work on improving the translation of negative sentences.
