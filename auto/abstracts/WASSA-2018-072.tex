Social media text written in Chinese communities contains mixed scripts including major text written with Chinese characters, an ideograph-based writing system, and some minor text using Latin letters, an alphabet-based writing system. This phenomenon is called writing systems change (WSCs). Past studies have shown that WSCs can be used to express emotions, particularly where the social and political environment is more conservative. However, because WSCs can break the syntax of the major text, it poses more challenges in NLP tasks like emotion classification. In this work, we present a novel deep learning based method to include WSCs as an effective feature for emotion analysis. The method first identifies all WSCs points. Representation of the major text is learned through an LSTM model whereas the presentation of the minority text is learned by a separate CNN.Emotions expressed in the minority text are further highlighted through an attention mechanism before emotion classification. It has proven to be significant that incorporating WSCs features in deep learning models can improve the performance which is valid by both F1-scores and p-value. It indicates that WSCs serve as an effective feature in emotion analysis of the social network.
