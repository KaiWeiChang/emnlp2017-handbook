Most work in machine reading focuses on question answering problems where the answer is directly expressed in the text to read. However, many real-world question answering problems require the reading of text not because it contains the literal answer, but because it contains a recipe to derive an answer together with the reader's background knowledge. One example is the task of interpreting regulations to answer ``Can I...?'' or ``Do I have to...?'' questions such as ``I am working in Canada. Do I have to carry on paying UK National Insurance?'' after reading a UK government website about this topic. This task requires both the interpretation of rules and the application of background knowledge. It is further complicated due to the fact that, in practice, most questions are underspecified, and a human assistant will regularly have to ask clarification questions such as ``How long have you been working abroad?'' when the answer cannot be directly derived from the question and text. In this paper, we formalise this task and develop a crowd-sourcing strategy to collect 37k task instances based on real-world rules and crowd-generated questions and scenarios. We analyse the challenges of this task and assess its difficulty by evaluating the performance of rule-based and machine-learning baselines. We observe promising results when no background knowledge is necessary, and substantial room for improvement whenever background knowledge is needed.
