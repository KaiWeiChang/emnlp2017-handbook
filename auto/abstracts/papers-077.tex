Developing a system that can automatically respond to a user's utterance has recently become a topic of research in natural language processing. However, most works on the topic take into account only a single preceding utterance to generate a response. Recent works demonstrate that the application of statistical machine translation (SMT) techniques towards monolingual dialogue setting, in which a response is treated as a translation of a stimulus, has a great potential, and we exploit the approach to tackle the context-dependent response generation task. We attempt to extract relevant and significant information from the wider contextual scope of the conversation, and incorporate it into the SMT techniques. We also discuss the advantages and limitations of this approach through our experimental results.
