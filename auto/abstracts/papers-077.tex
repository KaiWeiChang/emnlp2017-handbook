This paper addresses the automatic recognition of telicity, an aspectual notion. A telic event includes a natural endpoint (''she walked home''), while an atelic event does not (''she walked around''). Recognizing this difference is a prerequisite for temporal natural language understanding. In English, this classification task is difficult, as telicity is a covert linguistic category. In contrast, in Slavic languages, aspect is part of a verb's meaning and even available in machine-readable dictionaries. Our contributions are as follows. We successfully leverage additional silver standard training data in the form of projected annotations from parallel English-Czech data as well as context information, improving automatic telicity classification for English significantly compared to previous work. We also create a new data set of English texts manually annotated with telicity.
