This paper presents the Coptic Universal Dependency Treebank, the first dependency treebank within the Egyptian subfamily of the Afro-Asiatic languages. We discuss the composition of the corpus, challenges in adapting the UD annotation scheme to existing conventions for annotating Coptic, and evaluate inter-annotator agreement on UD annotation for the language. Some specific constructions are taken as a starting point for discussing several more general UD annotation guidelines, in particular for appositions, ambiguous passivization, incorporation and object-doubling.
