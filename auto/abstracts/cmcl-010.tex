While reading times are often used to measure working memory load, frequency effects (such as surprisal or n-gram frequencies) also have strong confounding effects on reading times. This work uses a naturalistic audio corpus with magnetoencephalographic (MEG) annotations to measure working memory load during sentence processing. Alpha oscillations in posterior regions of the brain have been found to correlate with working memory load in non-linguistic tasks (Jensen et al., 2002), and the present study extends these findings to working memory load caused by syntactic center embeddings. Moreover, this work finds that frequency effects in naturally-occurring stimuli do not significantly contribute to neural oscillations in any frequency band, which suggests that many modeling claims could be tested on this sort of data even without controlling for frequency effects.
