The existing studies in cross-language information retrieval (CLIR) mostly rely on general text representation models (e.g., vector space model or latent semantic analysis). These models are not optimized for the target retrieval task. In this paper, we follow the success of neural representation in natural language processing (NLP) and develop a novel text representation model based on adversarial learning, which seeks a task-specific embedding space for CLIR. Adversarial learning is implemented as an interplay between the generator process and the discriminator process. In order to adapt adversarial learning to CLIR, we design three constraints to direct representation learning, which are (1) a matching constraint capturing essential characteristics of cross-language ranking, (2) a translation constraint bridging language gaps, and (3) an adversarial constraint forcing both language and media invariant to be reached more efficiently and effectively. Through the joint exploitation of these constraints in an adversarial manner, the underlying cross-language semantics relevant to retrieval tasks are better preserved in the embedding space. Standard CLIR experiments show that our model significantly outperforms state-of-the-art continuous space models and is better than the strong machine translation baseline.
