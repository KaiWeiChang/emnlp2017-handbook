We present a machine learning analysis of eye-tracking data for the detection of mild cognitive impairment, a decline in cognitive abilities that is associated with an increased risk of developing dementia. We compare two experimental configurations (reading aloud versus reading silently), as well as two methods of combining information from the two trials (concatenation and merging). Additionally, we annotate the words being read with information about their frequency and syntactic category, and use these annotations to generate new features. Ultimately, we are able to distinguish between participants with and without cognitive impairment with up to 86\% accuracy.
