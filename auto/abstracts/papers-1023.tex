Recent work in NLP has attempted to deal with low-resource languages but still assumed a resource level that is not present for most languages, e.g., the availability of Wikipedia in the target language. We propose a simple method for cross-lingual named entity recognition (NER) that works well in settings with {\em very} minimal resources. Our approach makes use of a lexicon to ``translate'' annotated data available in one or several high resource language(s) into the target language, and learns a standard monolingual NER model there. Further, when Wikipedia is available in the target language, our method can enhance Wikipedia based methods to yield state-of-the-art NER results; we evaluate on 7 diverse languages, improving the state-of-the-art by an average of 5.5\% F1 points. With the minimal resources required, this is an extremely portable cross-lingual NER approach, as illustrated using a truly low-resource language, Uyghur.
