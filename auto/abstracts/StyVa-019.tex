While there is wide acknowledgement in NLP of the utility of document characterization by genre, it is quite difficult to determine a definitive set of features or even a comprehensive list of genres. This paper addresses both issues. First, with prototype semantics, we develop a hierarchical taxonomy of discourse functions. We implement the taxonomy by developing a new text genre corpus of contemporary German to perform a text based comparative register analysis. Second, we extract a host of style features, both deep and shallow, aiming beyond linguistically motivated features at situational correlates in texts. The feature sets are used for supervised text genre classification, on which our models achieve high accuracy. The combination of the corpus typology and feature sets allows us to characterize types of communicative purpose in a comparative setup, by qualitative interpretation of style feature loadings of a regularized discriminant analysis. Finally, to determine the dependence of genre on topics (which are arguably the distinguishing factor of sub-genre), we compare and combine our style models with Latent Dirichlet Allocation features across different corpus settings with unstable topics.
