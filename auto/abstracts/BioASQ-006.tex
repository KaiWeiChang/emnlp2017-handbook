There are millions of articles in PubMed database. To facilitate information retrieval, curators in the National Library of Medicine (NLM) assign a set of Medical Subject Headings (MeSH) to each article. MeSH is a hierarchically-organized vocabulary, containing about 28K different concepts, covering the fields from clinical medicine to information sciences. Several automatic MeSH indexing models have been developed to improve the time-consuming and financially expensive manual annotation, including the NLM official tool -- Medical Text Indexer, and the winner of BioASQ Task5a challenge -- DeepMeSH. However, these models are complex and not interpretable. We propose a novel end-to-end model, AttentionMeSH, which utilizes deep learning and attention mechanism to index MeSH terms to biomedical text. The attention mechanism enables the model to associate textual evidence with annotations, thus providing interpretability at the word level. The model also uses a novel masking mechanism to enhance accuracy and speed. In the final week of BioASQ Chanllenge Task6a, we ranked 2nd by average MiF using an on-construction model. After the contest, we achieve close to state-of-the-art MiF performance of {\textasciitilde}0.684 using our final model. Human evaluations show AttentionMeSH also provides high level of interpretability, achieving 0.90\@20 recall of all expert-labeled relevant words given an MeSH-article pair.
