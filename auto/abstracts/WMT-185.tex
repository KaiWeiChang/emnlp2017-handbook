Machine translation enables the automatic translation of textual documents between languages and can facilitate access to information only available in a given language for nonspeakers of this language, e.g. research results presented in scientific publications. In this paper, we provide an overview of the biomedical Translation shared task in the Workshop on Machine Translation (WMT) 2018, which specifically examined the performance of machine translation systems for biomedical texts. This year, we provided test sets of scientific publications from two sources (EDP and Medline) and for six language pairs (English with each of Chinese, French, German, Portuguese, Romanian and Spanish). We describe the development of the various test sets, the submissions that we received and the evaluations that we carried out. We obtained a total of 39 runs from six teams and some of this year's BLEU scores were somewhat higher that last year's, especially for teams that made use of biomedical resources or state-of-the-art MT algorithms (e.g. Transformer). Finally, our manual evaluation scored automatic translations higher than the reference translations for German and Spanish.
