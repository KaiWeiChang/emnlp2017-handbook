One of the weaknesses of current supervised word sense disambiguation (WSD) systems is that they only treat a word as a discrete entity. However, a continuous-space representation of words (word embeddings) can provide valuable information and thus improve generalization accuracy. Since word embeddings are typically obtained from unlabeled data using unsupervised methods, this method can be seen as a semi-supervised word sense disambiguation approach. This paper investigates two ways of incorporating word embeddings in a word sense disambiguation setting and evaluates these two methods on some SensEval/SemEval lexical sample and all-words tasks and also a domain-specific lexical sample task. The obtained results show that such representations consistently improve the accuracy of the selected supervised WSD system. Moreover, our experiments on a domain-specific dataset show that our supervised baseline system beats the best knowledge-based systems by a large margin.
