We introduce an advanced information extraction pipeline to automatically process very large collections of unstructured textual data for the purpose of investigative journalism. The pipeline serves as a new input processor for the upcoming major release of our New/s/leak 2.0 software,  which we develop in cooperation with a large German news organization. The use case is that journalists receive a large collection of files up to several Gigabytes containing unknown contents. Collections may originate either from official disclosures of documents, e.g. Freedom of Information Act requests, or unofficial data leaks. Our software prepares a visually-aided exploration of the collection to quickly learn about potential stories contained in the data. It is based on the automatic extraction of entities and their co-occurrence in documents. In contrast to comparable projects, we focus on the following three major requirements particularly serving the use case of investigative journalism in cross-border collaborations: 1) composition of multiple state-of-the-art NLP tools for entity extraction, 2) support of multi-lingual document sets up to 40 languages, 3) fast and easy-to-use extraction of full-text, metadata and entities from various file formats.
