BLEU is widely considered to be an informative metric for text-to-text generation, including Text Simplification (TS). TS includes both lexical and structural aspects. In this paper we show that BLEU is not suitable for the evaluation of sentence splitting, the major structural simplification operation. We manually compiled a sentence splitting gold standard corpus containing multiple structural paraphrases, and performed a correlation analysis with human judgments. We find low or no correlation between BLEU and the grammaticality and meaning preservation parameters where sentence splitting is involved. Moreover, BLEU often negatively correlates with simplicity, essentially penalizing simpler sentences.
