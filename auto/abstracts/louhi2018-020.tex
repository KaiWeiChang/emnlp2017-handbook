In this article, we describe the development of annotation guidelines for family history information in Norwegian clinical text. We make use of incrementally developed synthetic clinical text describing patients' family history relating to cases of cardiac disease and present a general methodology which integrates the synthetically produced clinical statements and guideline development. We analyze inter-annotator agreement based on the developed guidelines and present results from experiments aimed at evaluating the validity and applicability of the annotated corpus using machine learning techniques. The resulting annotated corpus contains 477 sentences and 6030 tokens. Both the annotation guidelines and the annotated corpus are made freely available and as such constitutes the first publicly available resource of Norwegian clinical text.
