In contrast to the older writing system of the 19th century, modern Hawaiian orthography employs characters for long vowels and glottal stops. These extra characters account for about one-third of the phonemes in Hawaiian, so including them makes a big difference to reading comprehension and pronunciation. However, transliterating between older and newer texts is a laborious task when performed manually. We introduce two related methods to help solve this transliteration problem automatically. One approach is implemented, end-to-end, using finite state transducers (FSTs). The other is a hybrid deep learning approach, which approximately composes an FST with a recurrent neural network language model.
