The framing of an action influences how we perceive its actor. We introduce connotation frames of power and agency, a pragmatic formalism organized using frame semantic representations, to model how different levels of power and agency are implicitly projected on actors through their actions. We use the new power and agency frames to measure the subtle, but prevalent, gender bias in the portrayal of modern film characters and provide insights that deviate from the well-known Bechdel test. Our contributions include an extended lexicon of connotation frames along with a web interface that provides a comprehensive analysis through the lens of connotation frames.
