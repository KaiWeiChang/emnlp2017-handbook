Punctuation is a strong indicator of syntactic structure, and parsers trained on text with punctuation often rely heavily on this signal. Punctuation is a diversion, however, since human language processing does not rely on punctuation to the same extent, and in informal texts, we therefore often leave out punctuation. We also use punctuation ungrammatically for emphatic or creative purposes, or simply by mistake. We show that (a) dependency parsers are sensitive to {\em both} absence of punctuation and to alternative uses; (b) neural parsers tend to be more sensitive than vintage parsers; (c) training neural parsers {\em without}{\textasciitilde}punctuation outperforms all out-of-the-box parsers across all scenarios where punctuation departs from standard  punctuation. Our main experiments are on synthetically corrupted data to study the effect of punctuation in isolation and avoid potential confounds, but we also show effects on out-of-domain data.
