Background: Verbal fluency tasks, which require producing as many words in response to a cue in a fixed time, are widely used within clinical neuropsychology and in neuropsychological research. Although semantic word lists can be elicited, typically only the number of words related to the cue is interpreted thus ignoring any structure in the word sequences. Automated language techniques can provide a much needed framework for extracting and charting useful semantic relations in healthy individuals and understanding how cortical disorders disrupt these knowledge structures and the retrieval of information from them. Methods: One minute, animal category verbal fluency tests from 150 participants consisting of healthy individuals, patients with schizophrenia, and patients with bipolar disorder were transcribed. We discuss the issues involved in building and evaluating semantic frameworks and developing robust features to analyze this data. Specifically we investigate a Latent Semantic Analysis (LSA) semantic space to obtain semantic features, such as pairwise semantic similarity and clusters. Results and Discussion: An in-depth analysis of the framework is presented, and then results from two measures based on LSA semantic similarity illustrate how these automated techniques provide additional, clinically useful information beyond word list cardinality.
