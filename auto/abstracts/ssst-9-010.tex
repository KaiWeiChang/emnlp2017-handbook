Morphological segmentation is an effective strategy for addressing difficulties caused by morphological complexity. In this study, we use an English-to-Arabic test bed to determine what steps and components of a phrase-based statistical machine translation pipeline benefit the most from segmenting the target language. We test several scenarios that differ primarily in when desegmentation is applied, showing that the most important criterion for success in segmentation is to allow the system to build target words from morphemes that span phrase boundaries. We also investigate the impact of segmented and unsegmented target language models (LMs) on translation quality. We show that an unsegmented LM is helpful according to BLEU score, but also leads to a drop in the overall usage of compositional morphology, bringing it to well below the amount observed in human references.
