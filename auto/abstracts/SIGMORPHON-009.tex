We investigate the lexical network properties of the large phoneme inventory Southern African language Mangetti Dune !Xung as it compares to English and other commonly-studied languages. Lexical networks are graphs in which nodes (words) are linked to their minimal pairs; global properties of these networks are believed to mediate lexical access in the minds of speakers. We show that the network properties of !Xung are within the range found in previously-studied languages. By simulating data (''pseudolexicons'') with varying levels of phonotactic structure, we find that the lexical network properties of !Xung diverge from previously-studied languages when fewer phonotactic constraints are retained. We conclude that lexical network properties are representative of an underlying cognitive structure which is necessary for efficient word retrieval and that the phonotactics of !Xung may be shaped by a selective pressure which preserves network properties within this cognitively useful range.
