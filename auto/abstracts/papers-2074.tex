Despite continuously improving performance, contemporary image captioning models are prone to ``hallucinating'' objects that are not actually in a scene. One problem is that standard metrics only measure similarity to ground truth captions and may not fully capture image relevance. In this work, we propose a new image relevance metric to evaluate current models with veridical visual labels and assess their rate of object hallucination. We analyze how captioning model architectures and learning objectives contribute to object hallucination, explore when hallucination is likely due to image misclassification or language priors, and assess how well current sentence metrics capture object hallucination. We investigate these questions on the standard image captioning benchmark, MSCOCO, using a diverse set of models. Our analysis yields several interesting findings, including that models which score best on standard sentence metrics do not always have lower hallucination and that models which hallucinate more tend to make errors driven by language priors.
