RNN language models have achieved state-of-the-art perplexity results and have proven useful in a suite of NLP tasks, but it is as yet unclear what syntactic generalizations they learn. Here we investigate whether state-of-the-art RNN language models represent long-distance filler---gap dependencies and constraints on them. Examining RNN behavior on experimentally controlled sentences designed to expose filler---gap dependencies, we show that RNNs can represent the relationship in multiple syntactic positions and over large spans of text. Furthermore, we show that RNNs learn a subset of the known restrictions on filler---gap dependencies, known as island constraints: RNNs show evidence for wh-islands, adjunct islands, and complex NP islands. These studies demonstrates that state-of-the-art RNN models are able to learn and generalize about empty syntactic positions.
