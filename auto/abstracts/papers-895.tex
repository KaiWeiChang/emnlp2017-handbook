The main goal of this paper is to develop out-of-domain (OOD) detection for dialog systems. We propose to use only in-domain sentences to build a generative adversarial network (GAN) of which the discriminator generates low scores for OOD sentences. To improve basic GANs, we apply feature matching loss in the discriminator, use domain-category analysis as an additional task in the discriminator, and remove the biases in the generator. Thereby, we reduce the huge effort of collecting OOD sentences for training OOD detection. For evaluation, we experimented OOD detection on a multi-domain dialog system. The experimental results showed the proposed method was most accurate compared to the existing methods.
