A number of recent works have proposed techniques for end-to-end learning of communication protocols among cooperative multi-agent populations, and have simultaneously found the emergence of grounded human-interpretable language in the protocols developed by the agents, learned without any human supervision! In this paper, using a Task \& Talk reference game between two agents as a testbed,  we present a sequence of `negative' results culminating in a `positive' one -- showing that while most agent-invented languages are effective (i.e. achieve near-perfect task rewards), they are decidedly not interpretable or compositional. In essence, we find that natural language does not emerge `naturally',despite the semblance of ease of natural-language-emergence that one may gather from recent literature. We discuss how it is possible to coax the invented languages to become more and more human-like and compositional by increasing restrictions on how two agents may communicate.
