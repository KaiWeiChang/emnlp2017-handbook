We present a very simple model for text quality assessment based on a deep convolutional neural network, where the only supervision required is one corpus of user-generated text of varying quality, and one contrasting text corpus of consistently high quality. Our model is able to provide local quality assessments in different parts of a text, which allows visual feedback about where potentially problematic parts of the text are located, as well as a way to evaluate which textual features are captured by our model. We evaluate our method on two corpora: a large corpus of manually graded student essays and a longitudinal corpus of language learner written production, and find that the text quality metric learned by our model is a fairly strong predictor of both essay grade and learner proficiency level.
