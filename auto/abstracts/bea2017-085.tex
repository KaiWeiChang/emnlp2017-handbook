Native Language Identification (NLI) is the task of automatically identifying the native language (L1) of an individual based on their language production in a learned language. It is typically framed as a classification task where the set of L1s is known a priori. Two previous shared tasks on NLI have been organized where the aim was to identify the L1 of learners of English based on essays (2013) and spoken responses (2016) they provided during a standardized assessment of academic English proficiency. The 2017 shared task combines the inputs from the two prior tasks for the first time. There are three tracks: NLI on the essay only, NLI on the spoken response only (based on a transcription of the response and i-vector acoustic features), and NLI using both responses. We believe this makes for a more interesting shared task while building on the methods and results from the previous two shared tasks. In this paper, we report the results of the shared task. A total of 19 teams competed across the three different sub-tasks. The fusion track showed that combining the written and spoken responses provides a large boost in prediction accuracy. Multiple classifier systems (e.g. ensembles and meta-classifiers) were the most effective in all tasks, with most based on traditional classifiers (e.g. SVMs) with lexical/syntactic features.
