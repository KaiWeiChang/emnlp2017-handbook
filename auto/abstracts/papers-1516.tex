Neural language models are a critical component of state-of-the-art systems for machine translation, summarization, audio transcription, and other tasks. These language models are almost universally autoregressive in nature, generating sentences one token at a time from left to right. This paper studies the influence of token generation order on model quality via a novel two-pass language model that produces partially-filled sentence ``templates'' and then fills  in missing tokens. We compare various strategies for structuring these two passes and observe a surprisingly large variation in model quality. We find the most  effective strategy generates function words in the first pass followed by content words in the second. We believe these experimental results justify a more extensive investigation of the generation order for neural language models.
