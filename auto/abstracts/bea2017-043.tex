Eye tracking studies from the past few decades have shaped the way we think of word complexity and cognitive load: words that are long, rare and ambiguous are more difficult to read. However, online processing techniques have been scarcely applied to investigating the reading difficulties of people with autism and what vocabulary is challenging for them. We present parallel gaze data obtained from adult readers with autism and a control group of neurotypical readers and show that the former required higher cognitive effort to comprehend the texts as evidenced by three gaze-based measures. We divide all words into four classes based on their viewing times for both groups and investigate the relationship between longer viewing times and word length, word frequency, and four cognitively-based measures (word concreteness, familiarity, age of acquisition and imagability).
