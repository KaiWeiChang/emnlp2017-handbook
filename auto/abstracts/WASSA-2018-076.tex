Sentiment Analysis has seen much progress in the past two decades. For the past few years, neural network approaches, primarily RNNs and CNNs, have been the most successful for this task. Recently, a new category of neural networks, self-attention networks (SANs), have been created which utilizes the attention mechanism as the basic building block. Self-attention networks have been shown to be effective for sequence modeling tasks, while having no recurrence or convolutions. In this work we explore the effectiveness of the SANs for sentiment analysis. We demonstrate that SANs are superior in performance to their RNN and CNN counterparts by comparing their classification accuracy on six datasets as well as their model characteristics such as training speed and memory consumption. Finally, we explore the effects of various SAN modifications such as multi-head attention as well as two methods of incorporating sequence position information into SANs.
