We propose an approach to detecting the rhetorical figure called chiasmus, which involves the repetition of a pair of words in reverse order, as in ``all for one, one for all''. Although repetitions of words are common in natural language, true instances of chiasmus are rare, and the question is therefore whether a computer can effectively distinguish a chiasmus from a random criss-cross pattern. We argue that chiasmus should be treated as a graded phenomenon, which leads to the design of an engine that extracts all criss-cross patterns and ranks them on a scale from prototypical chiasmi to less and less likely instances. Using an evaluation inspired by information retrieval, we demonstrate that our system achieves an average precision of 61\%. As a by-product of the evaluation we also construct the first annotated corpus of chiasmi.
