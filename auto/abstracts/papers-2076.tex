Grammatical error correction (GEC) systems deployed in language learning environments are expected to accurately correct errors in learners' writing. However, in practice, they often produce spurious corrections and fail to correct many errors, thereby misleading learners. This necessitates the estimation of the quality of output sentences produced by GEC systems so that instructors can selectively intervene and re-correct the sentences which are poorly corrected by the system and ensure that learners get accurate feedback. We propose the first neural approach to automatic quality estimation of GEC output sentences that does not employ any hand-crafted features. Our system is trained in a supervised manner on learner sentences and corresponding GEC system outputs with quality score labels computed using human-annotated references. Our neural quality estimation models for GEC show significant improvements over a strong feature-based baseline. We also show that a state-of-the-art GEC system can be improved when quality scores are used as features for re-ranking the N-best candidates.
