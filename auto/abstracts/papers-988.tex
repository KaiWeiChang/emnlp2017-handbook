Sarcasm is a pervasive phenomenon in social media, permitting the concise communication of meaning, affect and attitude. Concision requires wit to produce and wit to understand, which demands from each party knowledge of norms, context and a speaker's mindset. Insight into a speaker's psychological profile at the time of production is a valuable source of context for sarcasm detection. Using a neural architecture, we show significant gains in detection accuracy when knowledge of the speaker's mood at the time of production can be inferred. Our focus is on sarcasm detection on Twitter, and show that the mood exhibited by a speaker over tweets leading up to a new post is as useful a cue for sarcasm as the topical context of the post itself. The work opens the door to an empirical exploration not just of sarcasm in text but of the sarcastic state of mind.
