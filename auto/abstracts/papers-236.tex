One of the main obstacles for many Digital Humanities projects is the low data availability. Texts have to be digitized in an expensive and time consuming process whereas Optical Character Recognition (OCR) post-correction is one of the time-critical factors. At the example of OCR post-correction, we show the adaptation of a generic system to solve a specific problem with little data. The system accounts for a diversity of errors encountered in OCRed texts coming from different time periods in the domain of literature. We show that the combination of different approaches, such as e.g. Statistical Machine Translation and spell checking, with the help of a ranking mechanism tremendously improves over single-handed approaches. Since we consider the accessibility of the resulting tool as a crucial part of Digital Humanities collaborations, we describe the workflow we suggest for efficient text recognition and subsequent automatic and manual post-correction
