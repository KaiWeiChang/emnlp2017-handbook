The overall objective of 'social' dialogue systems is to support engaging, entertaining, and lengthy conversations on a wide variety of topics, including social chit-chat. Apart from raw dialogue data, user-provided ratings are the most common signal used to train such systems to produce engaging responses. In this paper we show that social dialogue systems can be trained more effectively by using unannotated data than by employing user-rating annotations, which are noisy, sparse, and expensive to collect. Using a dataset of real conversations collected in the 2017 Alexa Prize challenge, we developed a neural ranker for selecting `good' system responses to user utterances, i.e. responses which are likely to lead to long and engaging conversations. We show that (1) the neural ranker consistently outperforms several strong baselines when trained to optimise for user ratings; (2) when trained on larger amounts of unannotated data, using conversation length as the objective, the ranker performs better than one trained to maximise user ratings --- ultimately reaching a Precision\@1 of 0.87. This advance will make data collection for social conversational agents simpler and less expensive in the future.
