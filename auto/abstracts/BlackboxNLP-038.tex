We present a framework for analyzing what the state in RNNs remembers from its input embeddings. We compute the gradients of the states with respect to the input embeddings and decompose the gradient matrix with Singular Value Decomposition to analyze which directions in the embedding space are best transferred to the hidden state space, characterized by the largest singular values. We apply our approach to LSTM language models and investigate to what extent and for how long certain classes of words are remembered on average for a certain corpus. Additionally, the extent to which a specific property or relationship is remembered by the RNN can be tracked by comparing a vector characterizing that property with the direction(s) in embedding space that are best preserved in hidden state space.
