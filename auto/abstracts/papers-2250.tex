This paper studies semantic parsing for interlanguage (L2), taking semantic role labeling (SRL) as a case task and learner Chinese as a case language. We first manually annotate the semantic roles for a set of learner texts to derive a gold standard for automatic SRL. Based on the new data, we then evaluate three off-the-shelf SRL systems, i.e., the PCFGLA-parser-based, neural-parser-based and neural-syntax-agnostic systems, to gauge how successful SRL for learner Chinese can be. We find two non-obvious facts: 1) the L1-sentence-trained systems performs rather badly on the L2 data; 2) the performance drop from the L1 data to the L2 data of the two parser-based systems is much smaller, indicating the importance of syntactic parsing in SRL for interlanguages. Finally, the paper introduces a new agreement-based model to explore the semantic coherency information in the large-scale L2-L1 parallel data. We then show such information is very effective to enhance SRL for learner texts. Our model achieves an F-score of 72.06, which is a 2.02 point improvement over the best baseline.
