A recent study by Plank et al. (2016) found that LSTM-based PoS taggers considerably improve over the current state-of-the-art when evaluated on the corpora of the Universal Dependencies project that use a coarse-grained tagset. We replicate this study using a fresh collection of 27 corpora of 21 languages that are annotated with fine-grained tagsets of varying size. Our replication confirms the result in general, and we additionally find that the advantage of LSTMs is even bigger for larger tagsets. However, we also find that for the very large tagsets of morphologically rich languages, hand-crafted morphological lexicons are still necessary to reach state-of-the-art performance.
