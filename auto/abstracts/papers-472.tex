Most studies on human editing focus merely on syntactic revision operations, failing to capture the intentions behind revision changes, which are essential for facilitating the single and collaborative writing process. In this work, we develop in collaboration with Wikipedia editors a 13-category taxonomy of the semantic intention behind edits in Wikipedia articles. Using labeled article edits, we build a computational classifier of intentions that achieved a micro-averaged F1 score of 0.621. We use this model to investigate edit intention effectiveness: how different types of edits predict the retention of newcomers and changes in the quality of articles, two key concerns for Wikipedia today. Our analysis shows that the types of edits that users make in their first session predict their subsequent survival as Wikipedia editors, and articles in different stages need different types of edits.
