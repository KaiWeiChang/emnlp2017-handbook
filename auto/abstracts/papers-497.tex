More and more of the information available on the web is dialogic, and a significant portion of it takes place in online forum conversations about current social and political topics. We aim to develop tools to summarize what these conversations are about. What are the CENTRAL PROPOSITIONS associated with different stances on an issue; what are the abstract objects under discussion that are central to a speaker's argument? How can we recognize that two CENTRAL PROPOSITIONS realize the same FACET of the argument? We hypothesize that the CENTRAL PROPOSITIONS are exactly those arguments that people find most salient, and use human summarization as a probe for discovering them. We describe our corpus of human summaries of opinionated dialogs, then show how we can identify similar repeated arguments, and group them into FACETS across many discussions of a topic. We define a new task, ARGUMENT FACET SIMILARITY (AFS), and show that we can predict AFS with a .54 correlation score, versus an ngram system baseline of .39 and a semantic textual similarity system baseline of .45.
