Out-of-vocabulary words present a great challenge for Machine Translation. Recently various character-level compositional models were proposed to address this issue. In current research we incorporate two most popular neural architectures, namely LSTM and CNN, into hard- and soft-attentional models of translation for character-level representation of the source. We propose semantic and morphological intrinsic evaluation of encoder-level representations. Our analysis of the learned representations reveals that character-based LSTM  seems to be better at capturing morphological aspects compared to character-based CNN. We also show that hard-attentional model provides better character-level representations compared to vanilla one.
