In Visual Question Answering, most existing approaches adopt the pipeline of representing an image via pre-trained CNNs, and then using the uninterpretable CNN features in conjunction with the question to predict the answer. Although such end-to-end models might report promising performance, they rarely provide any insight, apart from the answer, into the VQA process. In this work, we propose to break up the end-to-end VQA into two steps: explaining and reasoning, in an attempt towards a more explainable VQA by shedding light on the intermediate results between these two steps. To that end, we first extract attributes and generate descriptions as explanations for an image. Next, a reasoning module utilizes these explanations in place of the image to infer an answer. The advantages of such a breakdown include: (1) the attributes and captions can reflect what the system extracts from the image, thus can provide some insights for the predicted answer; (2) these intermediate results can help identify the inabilities of  the image understanding or the answer inference part when the predicted answer is wrong. We conduct extensive experiments on a popular VQA dataset and our system achieves comparable performance with the baselines, yet with added benefits of explanability and the inherent ability to further improve with higher quality explanations.
