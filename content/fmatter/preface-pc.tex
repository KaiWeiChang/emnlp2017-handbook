\section{Message from the Program Committee Co-Chairs}
\setheaders%
    {Message from the Program Committee Co-Chairs}%
    {Message from the Program Committee Co-Chairs}
\thispagestyle{emptyheader}
%\renewcommand{\large}{\fontsize{9}{11}\selectfont}
% that's a hack to make this part nicely fill the pages

\setlength{\parskip}{.7ex}
%\setlength{\parindent}{0pt}

Welcome to the 2017 Conference on Empirical Methods in Natural Language Processing! This is an exciting year; we have received a new record-high in the number of submissions: 1,509 papers. After discounting early withdraws, duplicates, and other invalid submissions, we sent out 1,418 submissions (836 long papers, 582 short papers) to be reviewed by the program committee. Ultimately, 216 long papers (25.8\% acceptance rate) and 107 short papers (18.4\% acceptance rate) have been accepted for presentation, making a total of 323 papers and an overall acceptance rate of 22.8\%. 

This year’s technical program consists of three invited talks and 113 oral presentations and 219 poster presentations for the 323 long and short accepted papers as well as nine papers accepted to the Transactions of the Association for Computational Linguistics. To accommodate all the presentations in a compressed timeframe, we opted to have plenary sessions for the invited talks and the winners of the Best Paper Awards, while allotting three parallel oral sessions and thematically related poster sessions for all other presentations. We chose to have concurrent poster and oral sessions for several reasons. First, this is the preferred model of the majority (51.6\%) of participants who filled out the EMNLP 2015 post-conference survey. Second, this allows us to spread out the poster presentations across three days in smaller thematically related clusters. Finally, this maximises the number of acceptances for the high quality submissions we received; by having more poster sessions, we are able to maintain the acceptance rates at the previous year’s level despite an increase in submissions by 40\%. 

It would not have been possible to properly handle such a large number of submissions without the generous voluntary help from all the members of the program committee, which consists of 980 reviewers overseen by 51 area chairs. We continued last year’s experiment of defining twelve relatively broad topic areas and assigning multiple area chairs to facilitate consistent ranking of larger sets of papers. Most technical program decisions, from the selection of papers to the modes of presentation to the choice of outstanding papers, are primarily made in a bottom-up fashion: reviewers assessed and scored papers, made recommendations for oral vs poster decisions, and marked papers suitable for best paper awards; area chairs ensured the quality of assessments, encouraged discussions and assembled opinions into their own recommendations; finally, we construct the technical program, considering the recommendations from the area chairs while taking into account venue constraints and balance across areas. A new experimental feature of this year’s EMNLP reviewing process is the “meta review,” in which the area chairs briefly summarize the major discussions between the reviewers to give authors a more transparent view of the process. 

Per EMNLP tradition, awards are given to outstanding papers in three categories: Best Long Paper, Best Short Paper, and Best Resource Paper. The selection process is bottom-up:
based on the reviewers and area chairs’ recommendations, we nominated four papers for each category; we invited expert members to form a Best Papers committee for each category; each committee reviews the candidates and select the winners. The awarded papers will be presented at a special plenary session on the last day of the conference. 

We are extremely grateful that three amazing speakers have agreed to give invited talks at EMNLP. Nando de Freitas (Google Deepmind) will discuss simulated physical environments, and whether language would benefit from the development of such environments, and could contribute toward improving such environments and agents within them. Sharon Goldwater (University of Edinburgh) will describe work on developing unsupervised speech technology for those of the world's 7,000 or so languages not spoken in large rich countries. Dan Jurafsky (Stanford University) will talk about processing the language of policing to automatically measure linguistic aspects of the interaction from discourse factors like conversational structure to social factors like respect. 

\pagebreak
The conference would not have been possible without the support of various people inside and outside of the committee. In particular, we would like to thank:

\begin{itemize}
 
\item Martha Palmer, whose encouragement and advice as the general chair has been invaluable every step of the way;
\item Chris Callison-Burch, who has given us excellent advice and support in his capacity as the SIGDAT Secretary;
\item Priscilla Rasmussen, who always has the right answers;
\item Xavier Carreras and Kevin Duh, who generously shared their experiences as the chairs of EMNLP 2016; 
\item Anders Johannsen, who is lightning fast with website updates;
\item Our 51 area chairs:  David Bamman, Mohit Bansal, Roberto Basili, Chris Biemann, Jordan Boyd-Graber, Marine Carpuat, Joyce Chai, David Chiang, Jinho Choi, Jennifer Chu-Carroll, Trevor Cohn, Cristian Danescu-Niculescu-Mizil, Dipanjan Das, Hal Daume, Mona Diab, Mark Dredze, Jacob Eisenstein, Sanja Fidler, Alona Fyshe, Dan Gildea, Ed Grefenstette, Hannaneh Hajishirzi, Julia Hockenmaier, Kentaro Inui, Jing Jiang, Philipp Koehn, Mamoru Komachi, Anna Korhonen, Tom Kwiatkowski, Gina Levow, Bing Liu, Nitin Madnani, Mausam, Rada Mihalcea, Marie-Francine Moens, Saif M. Mohammad, Mari Ostendorf, Sameer Pradhan, Alexander Rush, Anoop Sarkar, William Schuler, Hinrich Schütze, Sameer Singh, Thamar Solorio, Vivek Srikumar, Amanda Stent, Tomek Strzalkowski, Mihai Surdeanu, Andreas Vlachos, Scott Wen-tau Yih, Zhang Yue;
\item The best papers award committee members: Chris Brew, Mike Collins, Kevin Duh, Adam Lopez, Ani Nenkova, Bonnie Webber, Luke Zettlemoyer;
\item Preethi Raghavan and Siddharth Patwardhan, the publications co-chairs and Joachim Bingel, the conference handbook chair;
\item Dirk Hovy and Anders Søgaard, the local arrangements co-chairs;
\item Rich Gerber and Paolo Gai at SoftConf.
 
\end{itemize}
Finally, we’d like to thank SIGDAT for the opportunity to serve as Program Co-Chairs of EMNLP 2017. It is an honor and a rewarding learning experience. We hope you will be as inspired by the technical program as we are. 

\vspace{3em}

\noindent EMNLP 2017 Program Co-Chairs \\
Rebecca Hwa, University of Pittsburgh\\
Sebastian Riedel, University College London
\index{Hwa, Rebecca}
\index{Riedel, Sebastian}
