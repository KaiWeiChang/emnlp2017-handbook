\begin{bio}

\textbf{Adam Lopez} develops computational models of natural language learning, understanding and generation
in people and machines, and his research focuses on basic scientific, mathematical, and
engineering problems related to these models. He is especially interested in models that handle
the diversity of morphological, syntactic, and semantic phenomena across the world�s languages.
He teaches undergraduate and master�s-level courses in natural language processing at the School
of Informatics at the University of Edinburgh, and previously in the Department of Computer
Science at the Johns Hopkins University. These courses have ranged in size from 10 to 200 students
from a variety of backgrounds, including computer science, mathematics, and linguistics. He
previously taught a popular course in machine translation at ESSLLI in 2010 and NASSLLI in
2012. He has given many tutorial lectures at outreach events for high school students, and once at
a poetry event for a local arts festival.

\textbf{Sorcha Gilroy} is currently a third year PhD student in the Edinburgh NLP group. Her research is
focused on probabilistic models of graphs. She has worked closely with Hyperedge Replacement
Grammars, Monadic Second-Order Logic and Directed Acyclic Graph Automata, which will all
be at least touched on in this course. She has given several talks on this topic in the past at the
Universities of Edinburgh, Amsterdam, Melbourne, the Information Sciences Institute, and the IT
University of Copenhagen. She has been a teaching assistant in courses on formal language theory,
machine translation, and machine learning at the University of Edinburgh; her responsibilities
included leading tutorial sessions for students. She once gave a public talk on machine translation
(using magnets) in front of the Scottish National Gallery as part of the UK�s Soapbox Science
festival.

\end{bio}

\begin{tutorial}
  {Graph Formalisms for Meaning Representations}
  {tutorial-final-002}
  {\daydateyear1, \tutorialmorningtime}
  {\TutLocNoIdea}

In this tutorial we will focus on Hyperedge Replacement Languages (HRL; Drewes et al. 1997), a context-free graph rewriting system. HRL are one of the most popular graph formalisms to be studied in NLP (Chiang et al., 2013; Peng et al., 2015; Bauer and Rambow, 2016). We will discuss HRL by formally defining them, studying several examples, discussing their properties, and providing exercises for the tutorial. While HRL have been used in NLP in the past, there is some speculation that they are more expressive than is necessary for graphs representing natural language (Drewes, 2017). Part of our own research has been exploring what restrictions of HRL could yield languages that are more useful for NLP and also those that have desirable properties for NLP models, such as being closed under intersection.

With that in mind, we also plan to discuss Regular Graph Languages (RGL; Courcelle 1991), a subfamily of HRL which are closed under intersection. The definition of RGL is relatively simple after being introduced to HRL. We do not plan on discussing any proofs of why RGL are also a subfamily of MSOL, as described in Gilroy et al. (2017b). We will briefly mention the other formalisms shown in Figure 1 such as MSOL and DAGAL but this will focus on their properties rather than any formal definitions.

\end{tutorial}