\begin{bio}
  {\bfseries Wei Lu} is an Assistant Professor at the Singapore University of Technology and Design (SUTD), directing the StatNLP research group. He received his Ph.D. from the National University of Singapore (NUS) in 2009. He visited CSAIL, Massachusetts Institute of Technology (MIT) in 2007-2008, and worked as a postdoctoral research associate at the University of Illinois at Urbana-Champaign in 2011-2013. His research interests include developing mathematical models and machine learning algorithms for solving natural language processing problems. He is particularly interested in semantic processing (in a broad sense). His papers appeared at venues such as ACL, EMNLP, NAACL, AAAI, and CIKM. He served as a program committee member for conferences such as ACL, EMNLP, NAACL, EACL, AAAI, IJCAI and NIPS, and is currently a member of the standing reviewer team for TACL. He served as an area co-chair for ACL 2016 and received the best paper award at EMNLP 2011.
\end{bio}

\begin{tutorial}
  {A Unified Framework for Structured Prediction: From Theory to Practice}
  {tutorial-final-006}
  {\daydateyear, \tutorialafternoontime}
  {\TutLocF}

Structured prediction is one of the most important topics in various fields, including machine learning, computer vision, natural language processing (NLP) and bioinformatics. In this tutorial, we present a novel framework that unifies various structured prediction models.

The hidden Markov model (HMM) and the probabilistic context-free grammars (PCFGs) are two classic generative models used for predicting outputs with linear-chain and tree structures, respectively. As HMM’s discriminative counterpart, the linear-chain conditional random fields (CRFs) (Lafferty et al., 2001) model was later proposed. Such a model was shown to yield good performance on standard NLP tasks such as information extraction. Several extensions to such a model were then proposed afterward, including the semi-Markov CRFs (Sarawagi and Cohen, 2004), tree CRFs (Cohn and Blunsom, 2005), as well as discriminative parsing models and their latent variable variants (Petrov and Klein, 2007). On the other hand, utilizing a slightly different loss function, one could arrive at the structured support vector machines (Tsochantaridis et al., 2004) and its latent variable variant (Yu and Joachims, 2009) as well. Furthermore, new models that integrate neural networks and graphical models, such as neural CRFs (Do et al., 2010) were also proposed.

In this tutorial, we will be discussing how such a wide spectrum of existing structured prediction models can all be implemented under a unified framework that involves some basic building blocks. Based on such a framework, we show how some seemingly complicated structured prediction models such as a semantic parsing model (Lu et al., 2008; Lu, 2014) can be implemented conveniently and quickly. Furthermore, we also show that the framework can be used to solve certain structured prediction problems that otherwise cannot be easily handled by conventional structured prediction models. Specifically, we show how to use such a framework to construct models that are capable of predicting non-conventional structures, such as overlapping structures (Lu and Roth, 2015; Muis and Lu, 2016a). We will also discuss how to make use of the framework to build other related models such as topic models and highlight its potential applications in some recent popular tasks (e.g., AMR parsing (Flanigan et al., 2014)).

The framework has been extensively used by our research group for developing various structured prediction models, including models for information extraction (Lu and Roth, 2015; Muis and Lu, 2016a; Jie et al., 2017), noun phrase chunking (Muis and Lu, 2016b), semantic parsing (Lu, 2015; Susanto and Lu, 2017), and sentiment analysis (Li and Lu, 2017). It is our hope that this tutorial will be helpful for many natural language processing researchers who are interested in designing their own structured prediction models rapidly. We also hope this tutorial allows researchers to strengthen their understandings on the connections between various structured prediction models, and that the open release of the framework will bring value to the NLP research community and enhance its overall productivity.

\end{tutorial} 
