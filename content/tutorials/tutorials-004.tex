\begin{bio}
\small

{\bfseries Svitlana Volkova} is a Ph.D.\ Candidate in Computer Science at the Center for
Language and Speech Processing, Johns Hopkins University. She works on
machine learning and natural language processing techniques for social
media predictive analytics. She develops batch and streaming (dynamic)
models for automatically inferring psycho-demographic profiles from
social media data streams, fine-grained emotion detection and
sentiment analysis for under-explored languages and dialects in
microblogs, effective interactive and iterative rationale annotation
via crowdsourcing.

{\bfseries Benjamin Van Durme} is the Chief Lead of Text Research at
the Human Language Technology Center of Excellence, and an Assistant
Research Professor at the Center for Language and Speech
Processing. He works on natural language processing (specifically
computational semantics), predictive analytics in social media and
streaming/randomized algorithms.

{\bfseries David Yarowsky} is a Professor at the Center for Language
and Speech Processing, Johns Hopkins University. His research
interests include natural language processing and spoken language
systems, machine translation, information retrieval, very large text
databases and machine learning. His research focuses on word sense
disambiguation, minimally supervised induction algorithms in NLP, and
multilingual natural language processing.

{\bfseries Yoram Bachrach} is a researcher in the Online Services and
Advertising group at Microsoft Research Cambridge UK. His research
area is artificial intelligence (AI), focusing on multi-agent systems
and computational game theory. Computational game theory combines the
theoretical foundations of economics and game theory with creative
solutions from AI and computer science.
\end{bio}

\begin{tutorial}
  {Social Media Predictive Analytics}
  {tutorial-final-019}
  {\daydateyear, \tutorialafternoontime}
  {\TutLocD}

  The recent explosion of social media services like Twitter, Google+
  and Facebook has led to an interest in social media predictive
  analytics – automatically inferring hidden information from the
  large amounts of freely available content. It has a number of
  applications, including: online targeted advertising, personalized
  marketing, large-scale passive polling and real-time live polling,
  personalized recommendation systems and search, and real-time
  healthcare analytics etc.

  In this tutorial, we will describe how to build a variety of social
  media predictive analytics for inferring latent user properties from
  a Twitter network including demographic traits, personality,
  interests, emotions and opinions etc. Our methods will address
  several important aspects of social media such as: dynamic,
  streaming nature of the data, multi-relationality in social
  networks, data collection and annotation biases, data and model
  sharing, generalization of the existing models, data drift, and
  scalability to other languages.

  We will start with an overview of the existing approaches for social
  media predictive analytics. We will describe the state-of-the-art
  static (batch) models and features. We will then present models for
  streaming (online) inference from single and multiple data streams;
  and formulate a latent attribute prediction task as a
  sequence-labeling problem. Finally, we present several techniques
  for dynamic (iterative) learning and prediction using active
  learning setup with rationale annotation and filtering. The tutorial
  will conclude with a practice session focusing on walk-through
  examples for predicting latent user properties e.g., political
  preferences, income, education level, life satisfaction and emotions
  emanating from user communications on Twitter.

\end{tutorial}
