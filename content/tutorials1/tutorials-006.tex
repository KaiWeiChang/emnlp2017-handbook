\begin{bio}
  {\bfseries Collin Baker} has been Project Manager of the FrameNet Project since 2000. His research interests include FrameNets in other languages (Loenneker-Rodman \& Baker 2009), aligning FrameNet to other lexical resources (Fellbaum \& Baker 2013, Ferrandez et al 2010), linking to ontologies and reasoning (Scheffczyk et al. 2010), and the frame semantics of metaphor.

  {\bfseries Nathan Schneider} has worked on a coarse-grained representation for lexical semantics (2014 dissertation at Carnegie Mellon University) and the design of the Abstract Meaning Representation (AMR; Banarescu et al.\ 2014). Nathan helped develop the leading open-source frame-semantic parser for English, SEMAFOR (Das et al.\ 2010, 2014) (\url{http://demo.ark.cs.cmu.edu/parse}), as well as a Python interface to the FrameNet lexicon (with Chuck Wooters) that is part of the NLTK suite.

  {\bfseries Miriam R. L. Petruck} received her PhD in Linguistics from the University of California, Berkeley. A key member of the team developing FrameNet almost since the project’s founding, her research interests include semantics, knowledge base development, grammar and lexis, lexical semantics, Frame Semantics and Construction Grammar.

  {\bfseries Michael Ellsworth}  has been involved with FrameNet for well over a decade. His chief focus is on semantic relations in FrameNet (Ruppenhofer et al.\ 2006), how they can be used for paraphrase (Ellsworth \& Janin 2007), and mapping to other resources (Scheffczyk et al.\ 2006, Ferrandez et al.\ 2010). Increasingly, he has examined the connection of FrameNet to syntax and the Constructicon (Torrent \& Ellsworth 2013, Ziem \& Ellsworth 2015), including in his pending dissertation on the constructions and frame semantics of emotion.
\end{bio}

\begin{tutorial}
  {Using FrameNet in NLP}
  {tutorial-final-022}
  {\daydateyear, \tutorialafternoontime}
  {\TutLocF}

The FrameNet lexical database (Fillmore \& Baker 2010, Ruppenhofer et
al. 2006, \url{http://framenet.icsi.berkeley.edu}), covers roughly
13,000 lexical units (word senses) for the core Engish lexicon,
associating them with roughly 1,200 fully defined semantic frames;
these frames and their roles cover the majority of event types in
everyday, non-specialist text, and they are documented with 200,000
manually annotated examples. This tutorial will teach attendees what
they need to know to start using the FrameNet lexical database as part
of an NLP system. We will cover the basics of Frame Semantics, explain
how the database was created, introduce the Python API and the state
of the art in automatic frame semantic role labeling systems; and we
will discuss FrameNet collaboration with commercial partners. Time
permitting, we will present new research on frames and annotation of
locative relations, as well as corresponding metaphorical uses, along
with information about how frame semantic roles can aid the
interpretation of metaphors.

\end{tutorial} 
