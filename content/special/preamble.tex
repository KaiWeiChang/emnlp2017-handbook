\usepackage[T1]{fontenc}
\usepackage{hyperref}
\usepackage[utf8]{inputenc}
\usepackage{textcomp}
\usepackage{scrextend}
\usepackage{graphicx}
\usepackage{color}
\definecolor{mygray}{gray}{0.75}
\usepackage{colortbl}
\usepackage{fancyhdr}
\usepackage[Bjornstrup]{fncychap}
\usepackage{longtable}
\usepackage{tabularx}
\usepackage{lscape}
\usepackage{array}
\usepackage{calc}
\usepackage{csquotes}
\usepackage[american]{babel}
\usepackage{multicol}
\usepackage{multirow}
\usepackage{times}
\usepackage{helvet}
\usepackage[maxnames=25,minnames=3,babel=hyphen]{biblatex}
\usepackage{bibentry}
\usepackage{setspace}
\usepackage{ifthen}
\usepackage{pstricks}
\usepackage{rotating}
\usepackage{makeidx}
\usepackage{marginnote}
\usepackage{ragged2e}
\usepackage{mathpazo}
\usepackage{graphbox}
\usepackage{booktabs}

%\providecommand{\BIBand}{and}

\newcommand{\leftheader}{}  
\newcommand{\rightheader}{} 
\pagestyle{fancy}
  % header spec
  %\renewcommand{\headrule}{{\color[rgb]{0.696,0,0}% 
  %\renewcommand{\headrule}{{\color[rgb]{0.132,0.125,.46}% 
  \renewcommand{\headrule}{{\color[HTML]{b12008}% 
    \hrule height 2pt width \headwidth}
    \vspace{1pt}%
    {\color{mygray}%
    \hrule height 1pt width \headwidth
  \vspace{-4pt}}}

  \fancyhf{}				       % clear header contents
  %\fancyhead[LE]{\textit{\nouppercase{\leftmark}}}
  %\fancyhead[RO]{\textit{\nouppercase{\rightmark}}} % define header contents
  \fancyhead[LE]{\textit{\nouppercase{\leftheader}}}
  \fancyhead[RO]{\textit{\nouppercase{\rightheader}}} % define header contents

  % footer spec
  \renewcommand{\footrule}{\hrule width \headwidth height 1mm\vskip\footruleskip}
  \renewcommand{\footruleskip}{0.5\normalbaselineskip}
  \fancyfoot[C]{\thepage}			 %define footer conten
%
%  \rfoot{\setlength{\unitlength}{1mm} % logo in right part of footer
%  \begin{picture}(0,0)
    %\put(-13,-10){\includegraphics[scale=0.3]{images/conf.jpg}}%
%    \put(-10,-5){\includegraphics[scale=0.2]{images/conf.jpg}}%
%  \end{picture}}

\makeatletter
\renewcommand\chapter{\if@openright\cleardoublepage\else\clearpage\fi %let headers/footers show on pages that start a chapter
                    \global\@topnum\z@
                    \@afterindentfalse
                    \secdef\@chapter\@schapter}

\renewcommand{\chaptermark}[1]{\markboth{#1}{}} % show chapter in header w/o numbering
\renewcommand{\sectionmark}[1]{\markright{#1}{}} % show section in headers w/o numbering

% redefine sections to have a horizontal rule and no numbering
\def\section{\@ifstar\unnumberedsection\numberedsection}
\def\numberedsection{\@ifnextchar[%]
  \numberedsectionwithtwoarguments\numberedsectionwithoneargument}
\def\unnumberedsection{\@ifnextchar[%]
  \unnumberedsectionwithtwoarguments\unnumberedsectionwithoneargument}
\def\numberedsectionwithoneargument#1{\numberedsectionwithtwoarguments[#1]{#1}}
\def\unnumberedsectionwithoneargument#1{\unnumberedsectionwithtwoarguments[#1]{#1}}
\def\numberedsectionwithtwoarguments[#1]#2{
  \ifhmode\par\fi
  \removelastskip
  \vskip 3ex\goodbreak
  \refstepcounter{section}%			   % increment counter
  \begingroup
  \noindent\begin{minipage}{\columnwidth}
  \leavevmode\Large\bfseries\raggedright
%  \thesection\  % leave out numbering
  #2 \par\nobreak
  \vskip -.5em
  \noindent\hrulefill\nobreak
  \end{minipage}
  \endgroup
  \vskip 1ex\nobreak
  \markright{#1}{} % add mark to right (secondary) header
  \addcontentsline{toc}{section}{%
%    \protect\numberline{\thesection}% % leave out number
    #1}%
  }
\def\unnumberedsectionwithtwoarguments[#1]#2{
  \ifhmode\par\fi
  \removelastskip
  \vskip 3ex\goodbreak
  \begingroup
  \noindent\begin{minipage}{\columnwidth}
  \leavevmode\Large\bfseries\raggedright
  #2\par\nobreak
  \vskip -.5em
  \noindent\hrulefill\nobreak
  \end{minipage}
  \endgroup
  \vskip 1ex\nobreak
  \markright{#1}{} % add mark to right (secondary) header
  }

% redefine subsections to have a horizontal rule and no numbering
\def\subsection{\@ifstar\unnumberedsubsection\numberedsubsection}
\def\numberedsubsection{\@ifnextchar[%]
  \numberedsubsectionwithtwoarguments\numberedsubsectionwithoneargument}
\def\unnumberedsubsection{\@ifnextchar[%]
  \unnumberedsubsectionwithtwoarguments\unnumberedsubsectionwithoneargument}
\def\numberedsubsectionwithoneargument#1{\numberedsubsectionwithtwoarguments[#1]{#1}}
\def\unnumberedsubsectionwithoneargument#1{\unnumberedsubsectionwithtwoarguments[#1]{#1}}
\def\numberedsubsectionwithtwoarguments[#1]#2{
  \ifhmode\par\fi
  \removelastskip
  \vskip 3ex\goodbreak
  \refstepcounter{subsection}%			   % increment counter
  \begingroup
  \noindent
  \leavevmode\normalsize\bfseries\raggedright
%  \thesubsection\  % leave out numbering
  #2 \par\nobreak
  \endgroup
  \vskip 1ex\nobreak
  \addcontentsline{toc}{subsection}{%
%    \protect\numberline{\thesubsection}% % leave out number
    #1}%
  }
\def\unnumberedsubsectionwithtwoarguments[#1]#2{
  \ifhmode\par\fi
  \removelastskip
  \vskip 3ex\goodbreak
  \begingroup
  \noindent
  \leavevmode\normalsize\bfseries\raggedright
  #2\par\nobreak
  \endgroup
  \vskip 1ex\nobreak
  }

\makeatother

% Clears to an even-numbered page
\def\clearevenpage{
     \clearpage 
     \ifodd\value{page} \hbox{}\newpage\fi
}

% Clears to the back-cover (for logos)
\def\cleartobackcover{
     \clearpage 
     \ifodd\value{page}\pagestyle{empty}\clearevenpage \else\hbox{}\cleardoublepage\pagestyle{empty}\clearevenpage\fi
}

%\renewcommand{\cleardoublepage}{\clearpage}

% Min: index
\makeindex

%\raggedbottom
%\setlength{\parindent}{0pt}
% Unicode issues
% \DeclareUnicodeCharacter{fi}{fi}

% Min: correct column widths
\newlength{\mycolumnwidth}
\setlength{\mycolumnwidth}{\columnwidth}
\addtolength{\mycolumnwidth}{-2ex}

\renewcommand{\textfraction}{.2}
\renewcommand{\bottomfraction}{.8}

\newenvironment{bio}
               {\begin{figure}[b]
                   \centerline{\rule{.5\linewidth}{.5pt}}\vspace{2ex}
                   \setlength{\parskip}{1ex}\setlength{\parindent}{0ex}}
               {\end{figure}}

\setlength{\parindent}{1em}

\fancypagestyle{emptyheader}
{
  \fancyhf{}
  \fancyfoot[C]{\thepage}
}

\newcommand{\setheaders}[2]{%
  \renewcommand{\leftheader}{#1}%
  \renewcommand{\rightheader}{#2}}
%%%%%%%%%%%%%%%%%%%%%%%%%%%%%%%%%%%%%%%%%%%%%%%%%%
