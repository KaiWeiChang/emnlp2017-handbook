\chapter[Anti-harassment policy]{Anti-harassment policy}
\thispagestyle{emptyheader}
\setheaders{}{}
The open exchange of ideas, the freedom of thought and expression, and respectful scientific debate are central to the aims and goals of the ACL. These require a community and an environment that recognizes the inherent worth of every person and group, that fosters dignity, understanding, and mutual respect, and that embraces diversity. For these reasons, ACL is dedicated to providing a harassment-free experience for all the members, as well as participants at our events and in our programs.

Harassment and hostile behavior are unwelcome at any ACL conference, associated event, or in ACL-affiliated on-line discussions. This includes: speech or behavior that intimidates, creates discomfort, or interferes with a person's participation or opportunity for participation in a conference or an event. We aim for ACL-related activities to be an environment where harassment in any form does not happen, including but not limited to: harassment based on race, gender, religion, age, color, appearance, national origin, ancestry, disability, sexual orientation, or gender identity. Harassment includes degrading verbal comments, deliberate intimidation, stalking, harassing photography or recording, inappropriate physical contact, and unwelcome sexual attention. The policy is not intended to inhibit challenging scientific debate, but rather to promote it through ensuring that all are welcome to participate in shared spirit of scientific inquiry.

It is the responsibility of the community as a whole to promote an inclusive and positive environment for our scholarly activities. In addition, anyone who experiences harassment or hostile behavior may contact any current member of the ACL Executive Committee or contact Priscilla Rasmussen, who is usually available at the registration desk during ACL conferences. Members of the executive committee will be instructed to keep any such contact in strict confidence, and those who approach the committee will be consulted before any actions are taken.

Approved by ACL Executive Committee in 2016.

The policy is also available from ACL's main page.
